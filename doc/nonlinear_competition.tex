\documentclass[12pt,a4paper]{article}
\usepackage{minionpro}
\usepackage{amsmath}
\usepackage{mathtools}
\newcommand{\ud}{\mathrm{d}}
\newcommand{\tr}{\mathrm{tr}}

\setcounter{secnumdepth}{0}

\begin{document}

\section{Logistic growth}

This is a classic analysis and nice and straightforward to think
about.  Everything else will build off this.  The continuous time
logistic equation is
%
\begin{equation}
  \label{eq:logistic}
  \frac{\ud N}{\ud t} = r N\left(1 - \frac{N}{K}\right)
\end{equation}
%
where $N$ is the density of the species, $r$ is the instantaneous
per-capita growth rate at zero density, and $K$ is the carrying
capacity.  There are two equilibria: $\hat N = 0$ and $\hat N = K$.

Taking the derivative of equation (\ref{eq:logistic}) with respect to
$N$ gives

\begin{equation}
  \label{eq:logistic-derivative}
  \frac{\ud}{\ud N}\left[\frac{\ud N}{\ud t}\right]
   = r - \frac{2 r N}{K}.
\end{equation}
%
When $\hat N = 0$, we have 
\begin{equation}
  \label{eq:logistic-derivative-trivial}
  \frac{\ud}{\ud N}\left[\frac{\ud N}{\ud t}\right]
   = r
\end{equation}
which is unstable for $r > 0$.  For the nontrivial equilibrium, $\hat
N = K$, we have
\begin{equation}
  \label{eq:logistic-derivative-nontrivial}
  \frac{\ud}{\ud N}\left[\frac{\ud N}{\ud t}\right]
   = -r
\end{equation}
which is stable for $r > 0$.

\section{Intraspecific-competition}

We can generalise equation (\ref{eq:logistic}) to include the effect
of intraspecific competition.  

\begin{equation}
  \label{eq:logistic-competition}
  \frac{\ud N}{\ud t} = r N\left(1 - \frac{N\alpha(N)}{K}\right)
\end{equation}

where $\alpha$ is a competition coefficient.  We still have an
equilibrium at $\hat N = 0$ but there are potentially more nontrivial
equilibria --- wherever where $N = K / \alpha(N)$.  Because we can
choose both $K$ and $\alpha$ we can define $\alpha(K) = 1$ and then we
retain an equilibrium at $\hat N = K$.

Taking the derivative of equation (\ref{eq:logistic-competition}) with
respect to $N$:
\begin{equation}
  \label{eq:logistic-competition-derivative}
  \frac{\ud}{\ud N}\left[\frac{\ud N}{\ud t}\right]
   = r - \frac{2 r N\alpha(N)}{K} - \frac{rN^2}{K}\alpha^\prime(N)
\end{equation}
where $\alpha^\prime(N) = \ud \alpha/\ud N$.

The conditions at the trivial equilibrium don't change because when $N
= 0$, equation (\ref{eq:logistic-competition-derivative}) reduces to
equation (\ref{eq:logistic-derivative-trivial}).  The nontrivial
equilibrium changes:
\begin{equation}
  \label{eq:logistic-competition-derivative-nontrivial}
  \frac{\ud}{\ud N}\left[\frac{\ud N}{\ud t}\right]
   = -r - rK\alpha^\prime(N) = -r(1 + K\alpha^\prime(N)).
\end{equation}
So the nontrivial equilibrium is stable if $r > 0$ and $K\alpha^\prime
> -1$ (other conditions apply for $r < 0$, but that's less
biologically relevant/interesting: we'd be stable there for $r < 0$
and $K\alpha^\prime < -1$).  Because $K \geq 0$, the sign of
$K\alpha^\prime$ is determined by $\alpha^\prime$.  It is negative
when the per-capita competitive effect decreases with increasing
density.  If it does de-stabilise the nontrivial equilibrium it must
open up another equilibrium between 0 and $K$ though, because the
trivial equilibrium is unconditionally unstable.

The nice thing about this approach is we can partition the stability
analysis into a density independent part and a density dependent
part.  What's not clear is if this will generally be easiest to work
with multiplicatively:
\begin{equation}
  \label{eq:logistic-comptition-partition-multiplicative}
  \widetilde \lambda =
  \overbrace{-r}^{\mathclap{\text{density-independent}}}
  \times
  \underbrace{(1 + K\alpha^\prime(N))}_{\text{density-dependent}}
\end{equation}
or additively:
\begin{equation}
  \label{eq:logistic-comptition-partition-additive}
  \widetilde \lambda =
  \overbrace{-r}^{\mathclap{\text{density-independent}}}
  +
  \underbrace{-rK\alpha^\prime(N)}_{\text{density-dependent}}
\end{equation}

\section{Linear interspecific competition}

For two species
\begin{equation}
  \label{eq:competitive-lv}
  \begin{split}
    \frac{\ud N_1}{\ud t} = &r_1 N_1
    \left(1 - \frac{N_1 \alpha_{11} + N_2 \alpha_{12}}{K_1}\right)\\
    \frac{\ud N_2}{\ud t} = &r_2 N_2
    \left(1 - \frac{N_1 \alpha_{21} + N_2 \alpha_{22}}{K_2}\right)
  \end{split}
\end{equation}
(this is the form used in Dieckmann and Dobeli, and is equivalent to
the form in Leimar with a change of variables that scales $r_i$ by
$K_i$).

There are four equilibria: $(0,0)$, $(K_1 / \alpha_{11}, 0)$, 
$(0, K_2 / \alpha_{22})$, and $(\hat N_1, \hat N_2)$, where

\begin{equation*}
  \begin{split}
    N_1 =& \frac{\alpha_{22} K_1 - \alpha_{12}K_2}
    {\alpha_{11}\alpha_{22} - \alpha_{12}\alpha_{21}}\\
    N_2 =& \frac{\alpha_{11} K_2 - \alpha_{21}K_1}
    {\alpha_{11}\alpha_{22} - \alpha_{12}\alpha_{21}}
  \end{split}
\end{equation*}
Note that the denominator of these equations is the determinant of a
matrix $[\alpha_{ij}]$.  We can define $\alpha_{11} = \alpha_{22} = 1$
(equivalent to rescaling $K_i$) to simplify things, but it will not
matter that much.

The Jacobian of this system is

\begin{equation}
  \label{eq:competitive-lv-jac}
  \mathbf{J} =
  \begin{bmatrix}
    r_1 - \frac{2 r_1 N_1 \alpha_{11}}{K_1} -
    \frac{r_1 N_2 \alpha_{12}}{K_1}&
    -\frac{r_1N_1\alpha_{12}}{K_1}\\
    -\frac{r_2N_2\alpha_{21}}{K_2}&
    r_2 - \frac{r_2N_1\alpha_{21}}{K_2} - \frac{2r_2N_2\alpha_{22}}{K_2}
  \end{bmatrix}
\end{equation}

We can consider the stability of the four equilibria in turn.
Starting with the trivial equilibrium $(0,0)$:

\begin{equation}
  \label{eq:competitive-lv-jac-trivial}
  \mathbf{J}_{(0,0)} =
  \begin{bmatrix}
    r_1 & 0 \\
    0   & r_2
  \end{bmatrix}
\end{equation}

This system is stable if $a_1 = -\tr(\mathbf{J}) > 0$ and $a_2 =
|\mathbf{J}| > 0$.  We have

\begin{equation*}
  \begin{split}
    a_1 =& -\tr(\mathbf{J}) = -(r_1 + r_2)\\
    a_2 =& |\mathbf{J}| = r_1 r_2
  \end{split}
\end{equation*}
%
If $r_1$ and $r_2$ are always positive then $a_1 < 0$ and $a_2 > 0$
and the system is always unstable.

To test the monomorphic nontrivial equilibrium $(K_1, 0)$ (which is
obviously very similar to the $(0, N_2)$ equilibrium but with
appropriate changes of indices):

\begin{equation}
  \label{eq:competitive-lv-jac-monomorphic}
  \mathbf{J}_{(K_1, 0)} =
  \begin{bmatrix}
    r_1 - \frac{2 r_1 N_1 \alpha_{11}}{K_1}&
    -\frac{r_1N_1\alpha_{12}}{K_1}\\
    0&
    r_2 - \frac{r_2N_1\alpha_{21}}{K_2}
  \end{bmatrix}
  =
  \begin{bmatrix}
    -r_1&
    -\frac{r_1N_1\alpha_{12}}{K_1}\\
    0&
    r_2\left(1  - \frac{K_1\alpha_{21}}{K_2\alpha_{11}}\right)
  \end{bmatrix}
\end{equation}
%
(with the last term simplifying a little with $\alpha_{11} = 1$)

This means that the stability criteria are:

\begin{equation*}
  \begin{split}
    a_1 =& -\tr(\mathbf{J}) = 
    r_1 - r_2\left(1  - \frac{K_1}{K_2}\alpha_{21}\right) > 0\\
    a_2 =& |\mathbf{J}| = 
    -r_1 r_2\left(1  - \frac{K_1}{K_2}\alpha_{21}\right) > 0
  \end{split}
\end{equation*}
%
If we assume that $r_i > 0$ then we can ignore the $r_1r_2$ term in
the $a_2$ condition and species 2 is incapable of invading if

\begin{equation*}
  \begin{split}
    \frac{K_1}{K_2}\alpha_{21} - 1 >&\, 0\\
    \alpha_{21} >&\, K_2 / K_1
  \end{split}
\end{equation*}
So if the competitive effect of species 1 on species 2 ($\alpha_{21}$)
is less than the ratio of the carrying capacities ($K_2/K_1$), then
species 2 can invade, with $a_2 < 0$.

The $a_1$ condition describes a different type of instability.
Species 2 can invade if

\begin{equation}
  \label{eq:invasion-a1}
  r_1 < r_2\left(1  - \frac{K_1}{K_2}\alpha_{21}\right)  
\end{equation}

If the term in parentheses is less than 0, then species 2 can
\emph{always} invade, even if this condition is not met, because
condition $a_1$ indicates instability (in this case, the rhs of the
$a_1$ condition will be negative).

Still have to do the $(\hat N_1, \hat N_2)$ equilibrium, but that's
fairly ugly to work out because the Jaobian retains all terms and the
actual $\hat N$ values are quite ugly.

\section{Density-dependent competition}

Here, let the $\alpha_{ij}$ terms be a function of $N_k$.  This
affects a number of things.  First, there may be other equilibria ---
we can't find them without saying things about the specific form of
$\alpha_{ij}(N_1, N_2)$ functions; there may be potentially many extra
equilibria.  However, the $(0,0)$, $(K_1 / \alpha_{11}, 0)$, $(0, K_2
/ \alpha_{22})$ equilibria still exist, and the $(\hat N_1, \hat N_2)$
equilibria exist at least where $\alpha_{ij}$ are not extremely
nonlinear, but possibly more generally too.  Note that in the above we
still define $\alpha_{11}(K_1, 0) = 1$ and $\alpha_{22}(0, K_2) = 1$
so that this model contains the previous model.

First, some terminology.  Let's define the Jacobian of this system as
$\mathbf{J} = \overline{\mathbf{J}} + \widetilde{\mathbf{J}}$, where
$\overline{\mathbf{J}}$ is the Jacobian from equation
(\ref{eq:competitive-lv-jac}) and $\widetilde{\mathbf{J}}$ is a set of
terms that only exist when $\alpha_{ij}$ is a function of $N_k$.
Recomputing the Jacobian of equation (\ref{eq:competitive-lv})
allowing the $\alpha_{ij}$ to depend on $N_k$ we get

\begin{equation}
  \label{eq:competitive-lv-jac2}
  \mathbf{J} = \overline{\mathbf{J}} +
  \begin{bmatrix}
    -\frac{r_1 N_1^2}{K_1}  \frac{\partial \alpha_{11}}{\partial N_1}
    -\frac{r_1 N_1N_2}{K_1} \frac{\partial \alpha_{12}}{\partial N_1}
    &
    -\frac{r_1 N_1^2}{K_1}  \frac{\partial \alpha_{11}}{\partial N_2}
    -\frac{r_1 N_1N_2}{K_1} \frac{\partial \alpha_{12}}{\partial N_2}
    \\
    -\frac{r_2 N_1N_2}{K_2}  \frac{\partial \alpha_{21}}{\partial N_1}
    -\frac{r_2 N_2^2}{K_2} \frac{\partial \alpha_{22}}{\partial N_1}
    &
    -\frac{r_2 N_1N_2}{K_2}  \frac{\partial \alpha_{21}}{\partial N_2}
    -\frac{r_2 N_2^2}{K_2} \frac{\partial \alpha_{22}}{\partial N_2}
  \end{bmatrix}
\end{equation}
where the term on the right is $\widetilde{\mathbf{J}}$, which can be
simplified to
\begin{equation*}
  \widetilde{\mathbf{J}} =
  \begin{bmatrix}
    -\frac{r_1 N_1}{K_1}\left(
      N_1 \frac{\partial \alpha_{11}}{\partial N_1} +
      N_2 \frac{\partial \alpha_{12}}{\partial N_1}\right)
    &
    -\frac{r_1 N_1}{K_1}\left(
      N_1\frac{\partial \alpha_{11}}{\partial N_2} +
      N_2\frac{\partial \alpha_{12}}{\partial N_2}\right)
    \\
    -\frac{r_2 N_2}{K_2}\left(
      N_1\frac{\partial \alpha_{21}}{\partial N_1} +
      N_2\frac{\partial \alpha_{22}}{\partial N_1}\right)
    &
    -\frac{r_2 N_2}{K_2}\left(
      N_1\frac{\partial \alpha_{21}}{\partial N_2} +
      N_2\frac{\partial \alpha_{22}}{\partial N_2}\right)
  \end{bmatrix}
\end{equation*}

First, note that in the trivial equilibrium $(0, 0)$ this density
dependence does not matter: $\widetilde{\mathbf{J}} = \mathbf{0}$ and
the conditions remain the same as before (that is, the system is
always invadeable if $(r_1, r_2) > 0$.

For the monomorphic equilibrium (again focussing arbitrarily in
nonzero density of species 1) $\widetilde{\mathbf{J}}$ reduces to
\begin{equation*}
  \widetilde{\mathbf{J}} =
  \begin{bmatrix}
    -\frac{r_1 N_1^2}{K_1} \frac{\partial \alpha_{11}}{\partial N_1}
    &
    -\frac{r_1 N_1^2}{K_1} \frac{\partial \alpha_{11}}{\partial N_2}
    \\0&0
  \end{bmatrix}
\end{equation*}
so

\begin{equation}
  \mathbf{J}_{(K_1, 0)} = 
  \overline{\mathbf{J}} + \widetilde{\mathbf{J}} = 
  \begin{bmatrix}
    -r_1&
    -\frac{r_1N_1\alpha_{12}}{K_1}\\
    0&
    r_2\left(1  - \frac{K_1\alpha_{21}}{K_2\alpha_{11}}\right)
  \end{bmatrix}
  +
  \begin{bmatrix}
    -\frac{r_1 N_1^2}{K_1} \frac{\partial \alpha_{11}}{\partial N_1}
    &
    -\frac{r_1 N_1^2}{K_1} \frac{\partial \alpha_{11}}{\partial N_2}
    \\0&0
  \end{bmatrix}
\end{equation}

The conditions for stability are
%
\begin{equation*}
  \begin{split}
    a_1 = - \tr(\mathbf{J}) =& 
    r_1\overbrace{\left(1 +
        K_1 \frac{\partial
          \alpha_{11}}{\partial N_1}
      \right)}^{\mathclap{\text{Extra due to density dependence}}} -
    r_2\left(1  - \frac{K_1}{K_2}\alpha_{21}\right)
    > 0\\
    a_2 = |\mathbf{J}| =&
    -r_1\underbrace{\left(1 + K_1
        \frac{\partial \alpha_{11}}{\partial N_1}\right)}%
    _{\mathclap{\text{Extra due to density dependence}}}
    r_2\left(1  - \frac{K_1}{K_2}\alpha_{21}\right)
    > 0
  \end{split}
\end{equation*}

It's surprising that while this equation involves $\alpha_{21}$ the
only terms for which the nonlinearity affects are the resident species
(so resident intraspecific competition nonlinearity matters, but we
only care about resident-invader effects through the $\alpha_{21}$
term evaluated at $(K_1, 0)$, and do not care about its slope).  The
way that $\partial\alpha_{11}/\partial N_1$ enters in here is the same
as equation (\ref{eq:logistic-comptition-partition-multiplicative}),
and the effects will be the same.  In particular, we'll only see a
sign change for $a_2$ when $K_1\partial\alpha_{11}/\partial N_1 < -1$.
This implies that the per-capita intraspecific competition function
$\alpha_{11}$ strongly decreases with population size, and makes no
reference to the invading population: this is probably the system
moving to an additional stable equilibrium $0 < N_1 < K_1$.

The $a_1$ condition is more interesting though; we modify equation
(\ref{eq:invasion-a1}) to get the conditions for \textit{invasion} by
species 2:

\begin{equation}
  \label{eq:invasion-a1-v2}
  r_1\overbrace{\left(1 +
      K_1 \frac{\partial \alpha_{11}}{\partial N_1}
    \right)}^{\mathclap{\text{Extra due to density dependence}}} < r_2\left(1  - \frac{K_1}{K_2}\alpha_{21}\right)  
\end{equation}

So if $\frac{\partial \alpha_{11}}{\partial N_1}$ is negative then
species 2 can invade over a wider range of parameters than for when
competition is density independent.

The above might be slightly easier to process, though require
re-deriving, using the Leimar approach to parametrising the model.
This affects only $r$ and $K$, so might wash through fairly easily.

\end{document}

%%% Local Variables: 
%%% mode: latex
%%% TeX-PDF-mode: t
%%% End: 
