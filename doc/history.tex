%MW possibly better to leave \Rstar\ and the nature of coexistence made possible till it is introduced for analysis?
\documentclass[a4paper,11pt]{article}
\usepackage[osf]{mathpazo}
\usepackage{ms}
\usepackage{natbib}
\usepackage{graphicx}
\usepackage{caption}
\usepackage[labelfont=bf]{caption} % make label for figure bold

% Allow referencing into the supporting information, once that exists.
\IfFileExists{./competition-kernels-sm.tex}{%
  \usepackage{xr}%
  \externaldocument{competition-kernels-sm}}{}

\title{Competition kernels: shapes and consequences}
\author{}
\date{}
\affiliation{}
\runninghead{}
\keywords{}

\usepackage{color}

\newcommand{\ud}{\ensuremath{\mathrm{d}}}
\newcommand{\sign}{\mathop{\mathrm{sign}}\nolimits}
\newcommand{\Rstar}{\ensuremath{R^*}}
\newcommand{\plant}{{\tt plant}}
\newcommand{\hmat}{\ensuremath{h_{\text{mat}}}}
\newcommand{\TODO}{{\color{red}\sc todo}}

\newcommand{\figRstar}{3}
\newcommand{\figPlantHmat}{4}
\newcommand{\figPlantLma}{5}


\begin{document}

%MW version starting here
Central to community ecology has been the metaphor of species packing. The community is seen as fitted into a volume of niche space. Each species that is a potential community member has a fundamental niche somewhere in the volume. The fundamental niche is compressed by competition into a realised niche, as species are packed in alongside each other (Figure \ref{fig:competition-kernels} for one-dimensional version found in [almost all] textbooks). Roots of the species packing metaphor trace to the competitive exclusion principle \cite{Lotka-1925, Volterra-1926, Gause-1934} whereby if species have identical niches or compete for a single resource, all but one are competitively excluded. \citet{Hutchinson-1957} added the n-dimensional hyper-volume. The  competitive exclusion principle was converted [inverted? made more quantitative?] into the question how dissimilar two species need to be in order to coexist. This question included ``limiting similarity'' \citep{MacArthur-1967}, and also ``character displacement'' \citep{Brown-1956}, the evolution of wider trait differences between species where they coexist. By the 1970s, a new generation of textbooks \citep{Whittaker-1970,Krebs-1972,Ricklefs-1973,pianka-1974} built their account of community ecology around three alternative ways of getting more species into a community: larger total niche space, more overlap between adjacent niches, or narrower niches for each species [could potentially give page or diagram references here?]. Often, these alternatives were discussed as hypotheses for why there are more species in the tropics compared to the temperate zone.

%MW Haven't yet said anything about the empirical literature looking for overdispersion of traits in communities -- is it important to our Intro, and if so where should it go? Maybe leave till Discussion?

In 1960s theory and 1970s textbooks, niches were typically thought of as utilization functions along a continuous resource axis. The resource in question would be depletable or pre-emptable, in order for overlap of utilization functions to translate into competition. Pairwise competition coefficients (such as alphas in Lotka-Volterra competition) could be calculated from overlap of utilizations functions. If a trait was used as a niche index, it was chosen on grounds of mapping tidily on to a depletable resource, such as beak size for seed-eating finches. Beginning in the 1980s, two different theory strands represented traits more explicitly. One was theory for character (= trait) displacement \citep[e.g.,][]{Taper-1985, Case-2000, Goldberg-2006}. This flowed on into game theoretical and eventually adaptive-dynamic modeling of trait divergence \citep{Brown-1987,Geritz-1998,Geritz-1999,Dieckmann-1999,Abrams-2001}. The other was Tilman's \citeyear{Tilman-1982} \Rstar\ model, where the ``trait'' was the level to which a species could deplete a resource while still sustaining its population. The resource was homogeneous, such as dissolved nitrate in a water body, and this was a distinct break from the formulation of a resource with such as seeds with a continuously varying property such as seed size. Tilman's \citeyear{Tilman-1982} model would have a single winner on a single resource (the species with lowest \Rstar), but given two or more resources multiple species could coexist on different ratios between resources.

Competition kernels are best thought of within the adaptive-dynamics tradition of fitness landscapes. Species exist, coexist and evolve on a landscape where the axes are evolvable traits. A competition kernel represents the impact of a species on this landscape -- the drawdown of fitness across the landscape as it relates to the trait-values both of the species exerting a competitive effect and of the species experiencing the effect. Competition kernels are absolutely central to the question how species interact in competitive communities. And yet, surprisingly little is known about alternative possible shapes for competition kernels and their consequences for community assembly.

By far the most common type of utilization curve or competition kernel in textbooks
and in theoretical literature is that they are unimodal, decreasing with trait-distance away from the species that is exerting the competitive effect. For example \citep{Birand-2014} wrote ``we assume that the competition function is Gaussian, which is typically assumed in much of the literature''.
%MW If we wanted to abbreviate the history, we could maybe at this point go straight to the theory that shows you don't necessarily get species-spacing from Gaussian kernels

The species packing metaphor has persisted up to the present day as the mainstream approach to community ecology. (Indeed,  \citet{Krebs-2013}, \citet{Ricklefs-1999} and \citet{Pianka-1999} continue also.) Yet in parallel, three other strands of thinking had already been established by the 1970s.

Terrestrial plant ecology had always found it difficult to accept the idea that different species can only coexist by using different resources, since plant species clearly all use the same sources of light, soil water and mineral nutrients as each other, with only limited opportunities for differentiation by soil depth and by chemical forms of nutrients. One possibility arises from species subdividing a successional-time niche axis in landscapes envisaged as disturbance mosaics \citep{Connell-1978}. If plant species make most contribution to their population increase in different years following disturbance, then they certainly are depleting different light and water from each other, and quite possibly different soil nutrients \citep{Huston-1987, Kohyama-1993}.
%[MW -- these refs imported from Rich's notes -- need to check whether they're appropriate for this point].

In a more radical break from subdividing continuous niche axes, models arose that
visualised communities as a chessboard of living sites, each able to be occupied by one sessile individual \citep{Yodzis-1978}. Coexistence could arise by colonization-competition trade-offs \citep{Levins-1969,Levins-1971}. Where competition is size-asymmetric as in plants competing for light (\citep{Harper-1977,Weiner-1990}, or where there is advantage to individuals established in territories, as in some coral reef fish, then chance effects in the occupation of vacant living sites can slow competitive exclusion by ``lottery'' \citep{Sale-1977}( or ``storage effect'' \citep{Chesson-1981} processes. There have been several theory-lineages within this ``sessile dynamics framework'' \citep{Fagerstrom-1997}, and ``neutral theory'' \citep{Hubbell-2001} also envisions the world as a (very large) chessboard of living sites on which stochastic processes occur.

Another strand of thinking parallel to the species packing metaphor has been ecological strategy theory 
\citep[e.g.][]{Grime-1974, Grime-1977, Grime-1979, Grime-1988, Southwood-1977, 
Greenslade-1983}. In the best-developed ``CSR'' version of this, species were arrayed on a triangle with competition-winning, stress-tolerating and ruderal strategies at the corners. Stress-tolerating included persistence under competition. Predictions arose about the strategy-mixture expected in different physical settings and at different stages of succession \citep{Grime-1979,Grime-2001}. Then during the 1990s the suggestion was made to sidestep continuing disputes about the definition of competitiveness by focusing instead on measurable traits and using them as strategy axes \citep{Westoby-1998, Weiher-1999, Westoby-2002}. This suggestion made possible studies of coordination and trade-offs between traits on a worldwide scale \citep[e.g.][]{Wright-2004, Zanne-2010, Cornwell-2014}. However, this ``trait ecology'' has yet to return successfully to the question how species fit together in communities. By being registered on trait axes, competition kernels have the potential to link this world-scale empirical trait ecology back to questions of species assembly into communities.

A third strand of thinking, already established during the 1970s in parallel with the species packing metaphor, was theory showing that with Gaussian kernels along a single niche axis, there is in fact no limit to similarity \citep{May-1972, Abrams-1975, Abrams-1983}. Competitive exclusion can become slow because two competitors are similar in their resource use as well as because they are different. ``Continuous coexistence'' is a theoretically possible outcome, with resources occupied by many species spaced from each other in resource use by very small amounts. Although this strand of theory might have been expected to undermine the mainstream approach to community ecology via the species packing metaphor, in actuality it did not have the effect of dislodging species packing from standard textbooks, probably because it did not really offer an alternative program for connecting theory to field measurement.  After 2000, this line of theory began to explore more systematically the consequences of possible kernel-shapes for community assembly \citep{Scheffer-2006}, other literature, \citep{Leimar-2013}.

In summary, the current situation is that it is clear competition kernels on fitness landscapes must be the key to understanding community assembly, and clear also that symmetric unimodal or Gaussian kernels are realistic for only a minority of real-world communities. At the same time, the menagerie of possible competition kernels has only begun to be catalogued. Much less have the implications of different kernels for community assembly been worked through. The literature up to the present has mainly been focused on Gaussian kernels, perhaps because they were mistakenly thought to be prevalent, or out of fascination with the ``continuous coexistence'' they can produce. Surprisingly little attention has been paid to the question what shapes are actually likely for competition kernels, given known mechanisms for resource depletion in relation to traits. We argue that clarifying plausible shapes for competition kernels and their consequences will be an essential foundation before a robust community ecology can be constructed.

\end{document}