\documentclass[12pt,a4paper]{article}
\usepackage{amsmath}
\usepackage{fullpage}
\usepackage[osf]{mathpazo}
\newcommand{\ud}{\ensuremath{\mathrm{d}}}
\newcommand{\Rstar}{\ensuremath{\mathrm{R}^*}}
\newcommand{\Rs}{\ensuremath{R^*}}
\newcommand{\Ns}{\ensuremath{N^*}}
\newcommand{\Rhat}{\ensuremath{\hat R}}
\newcommand{\Nhat}{\ensuremath{\hat N}}

\begin{document}

Fitness is defined as the \emph{per capita rate of population
  increase} -- that is, the number of offspring per individual per
unit time and so has units of $N_I / (N_I t)$ where the subscript $I$
indicates invader.  We will generally consider this evaluated at $N_I
= 0$.

Taking the derivative with respect to resident density $N_R$,

\begin{equation*}
  \frac{\ud}{\ud N_R}\left[ \frac{\ud N_I}{\ud t} \right]
\end{equation*}

gives a quantity that has units $\frac{N_I}{N_I t}\frac{1}{N_R}$.

One of the scalings was dividing by the fitness in an empty
environment $w_0$.  This is basically the maximum value of $\ud N_I/
\ud t$, and it makes the fitness dimensionless.  Because $w_0$ does
not depend on $N_R$

\begin{equation*}
  \frac{\ud}{\ud N_R} \left[\frac{1}{w_0}\frac{\ud N_I}{\ud t}\right]
  =
  \frac{1}{w_0}\frac{\ud}{\ud N_R}\left[\frac{\ud N_I}{\ud t}\right]
\end{equation*}

Units wise, this converts the fitness measure from being individuals
per individual per time to fractions of maximum individuals per
individual per time, and the competition measure to being per invader
per resident per time.

The other scaling was multiplying back through by the density of
residents so we have

\begin{equation*}
  \frac{1}{w_0}\frac{\ud}{\ud N_R}\left[\frac{\ud N_I}{\ud t}\right] N_R
\end{equation*}

which is dimensionless

\begin{equation*}
  \underbrace{\frac{N_I}{N_I t}}_{\text{fitness}}
  \times
  \overbrace{\frac{N_I t}{N_I}}^{1/\text{max fitness}}
  \times
  \underbrace{\frac{1}{N_R}}_{\frac{\ud}{\ud N_R}}
  \times
  \overbrace{\frac{N_R}{1}}^{N_R}
\end{equation*}

I don't know if that's necessarily a good thing, but it does seem that
it might make it easier to compare across different models, or
different parameter combinations.  In particular, getting time out of
there is nice because you can scale ecological interactions up and
down in speed and get the same answer.

The dependence on $N_R$ is probably only a problem when considering
different resident densities.  In particular, in a particular
scenario, comparing across invader traits while holding a particular
resident community constant will not depend on the $N_R$ term.

Units of the competition kernel appear dimensionless in Dieckmann and
Dobelli, because it appears as $\ud N / \ud t = r N(1 - (N + \alpha
N_R)/K)$.  It's dimensionless in usual Lotka-Volterra models (because
it appears as $K_i - N_i - \alpha_j N_j$, $\alpha_j$ must be
dimensionless).  Also in Firbank and Watkinson (eq 9). 

But it appears to have units $1/t$ in Kisdi because it appears as $r -
\alpha(x_i - x_j)N_j$.  Unless we let $K = 1/r$ to get $r(1 -
\alpha(x_i - x_j)N_j / K)$ in which case it's dimensionless again, but
does appear scaled by time.

\end{document}

%%% Local Variables: 
%%% mode: latex
%%% TeX-PDF-mode: t
%%% End: 
