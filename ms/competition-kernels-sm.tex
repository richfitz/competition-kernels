\documentclass[10pt,twoside]{article}

\usepackage{ms/suppmat}
\usepackage{listings}
\usepackage{amsmath}

\usepackage{graphicx}

\usepackage{color}

% this doesn't work:
% \IfFileExists{./competition-kernels.tex}{%
%   \usepackage{xr}%
%   \externaldocument{competition-kernels}}{}

% We will generate all images so they have a width \maxwidth. This means
% that they will get their normal width if they fit onto the page, but
% are scaled down if they would overflow the margins.
\makeatletter
\def\maxwidth{\ifdim\Gin@nat@width>\linewidth\linewidth\else\Gin@nat@width\fi}
\def\maxheight{\ifdim\Gin@nat@height>\textheight\textheight\else\Gin@nat@height\fi}
\renewcommand{\thefigure}{A\@arabic\c@figure}

\makeatother
\setkeys{Gin}{width=\maxwidth,height=\maxheight,keepaspectratio}

\title{Emergent shapes of trait-based competition functions from resource-based models: A Gaussian is not normal in plant communities}
\date{}
\author{Daniel S. Falster, Georges Kunstler, Richard G. FitzJohn, Mark Westoby}
\runninghead{}
\address{}
\emailaddress{}


% \usepackage{natbib}

\begin{document}

\maketitle

\setcounter{secnumdepth}{1}

\renewcommand{\thefigure}{A\arabic{figure}}
\setcounter{figure}{0}

\begin{figure}[ht]
 \centering
	\includegraphics[width=1.0\textwidth]{ms/figures/rstar_components_U_1}
 \caption{{\bf Shape of competition function embedded within the consumer-resource model for a trait determining the relative amounts of the two essential resources required for growth, with constant symmetric resource content.} Panel descriptions as in Fig. 2. Parameters for the resource content, $C = \{2, 2\}$. Resident trait values are: (c,d) 0.6, (e,f) 0.5. All other parameters are given in the Methods and also in the code.}
 \label{figS:Rstar_U1}
\end{figure}

\newpage
\begin{figure}[ht]
 \centering
	\includegraphics[width=1.0\textwidth]{ms/figures/rstar_components_U_2}
 \caption{{\bf Shape of competition function embedded within the consumer-resource model for a trait determining the relative amounts of the two essential resources required for growth, with constant asymmetric resource content.} Panel descriptions as in Fig. 2. Parameters for the resource content, $C = \{0.3, 0.7\}$. Resident trait values are: (c,d) 0.7, (e,f) 0.9. All other parameters are given in the Methods and also in the code.}
 \label{figS:Rstar_U2}
\end{figure}

\newpage
\begin{figure}[ht]
 \centering
 \includegraphics[width=1.0\textwidth]{ms/figures/rstar_components_UC_2}
 \caption{{\bf Shape of competition function embedded within the consumer-resource model for a trait determining the relative amounts of the two essential resources required for growth, with resource content equal to resource half saturation determined by the trait such as $C_{ki} = U_{ki}$ with asymmetrical resource supply $S = \{0.7, 0.3\}$.} Panel descriptions as in Fig. 2. Resident trait values are: (c,d) 0.9, (e,f) 0.7. All other parameters are given in the Methods and also in the code.} 
 \label{fig:Srstar_components_UC2}
\end{figure}


\end{document}

%%% Local Variables:
%%% mode: latex
%%% TeX-master: t
%%% TeX-PDF-mode: t
%%% End:
