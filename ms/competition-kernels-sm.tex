\documentclass[10pt,twoside]{article}

\usepackage{suppmat}
\usepackage{listings}

\usepackage{graphicx}

\usepackage{color}

% this doesn't work:
% \IfFileExists{./competition-kernels.tex}{%
%   \usepackage{xr}%
%   \externaldocument{competition-kernels}}{}

\newcommand{\ud}{\ensuremath{\mathrm{d}}}
\newcommand{\sign}{\mathop{\mathrm{sign}}\nolimits}
\newcommand{\Rstar}{\ensuremath{R^*}}
\newcommand{\plant}{{\tt plant}}
\newcommand{\hmat}{\ensuremath{h_{\text{mat}}}}
\newcommand{\TODO}{{\color{red}\sc todo}}

\newcommand{\figRstar}{3}
\newcommand{\figPlantHmat}{4}
\newcommand{\figPlantLma}{5}


% We will generate all images so they have a width \maxwidth. This means
% that they will get their normal width if they fit onto the page, but
% are scaled down if they would overflow the margins.
\makeatletter
\def\maxwidth{\ifdim\Gin@nat@width>\linewidth\linewidth\else\Gin@nat@width\fi}
\def\maxheight{\ifdim\Gin@nat@height>\textheight\textheight\else\Gin@nat@height\fi}
\renewcommand{\thefigure}{S\@arabic\c@figure}

\makeatother
\setkeys{Gin}{width=\maxwidth,height=\maxheight,keepaspectratio}

\title{competition kernels: shapes and consequences}
\date{}

% \usepackage{natbib}

\begin{document}

\maketitle

\setcounter{secnumdepth}{1}

\begin{figure}[h]
 \centering
 \includegraphics{figures/sm_rstar_components1}
 \caption{Components used to estimate competition functions shown in Figs.
 {\figRstar}b-c for \Rstar\ model with symmetric resource requirements.
  Panels show: (a) the fitness of a rare invader growing without competition,
  (b) the equilibrium population density of invader when growing in monoculture,
  (c1, c2) the fitness of a rare invader growing in competition with an established
  resident, whose location is indicated by dashed line, and (d1,d2) derived competition
  functions for communities in panels  c1 \& c2. Parameters as in Fig.
  {\figRstar}a-c.}
  \label{fig:sm_rstar_components1}
\end{figure}

\begin{figure}[h]
 \centering
 \includegraphics{figures/sm_rstar_components2}
 \caption{Components used to estimate competition functions shown in Figs.
 {\figRstar}e-f for \Rstar\ model with asymmetric resource requirements.
  Panels show: (a) the fitness of a rare invader growing without competition,
  (b) the equilibrium population density of invader when growing in monoculture,
  (c1, c2) the fitness of a rare invader growing in competition with an established
  resident, whose location is indicated by dashed line, and (d1,d2) derived competition
  functions for communities in panels c1 \& c2. Parameters as in Fig.
  {\figRstar}d-f.}
  \label{fig:sm_rstar_components2}
\end{figure}

\begin{figure}[h]
 \centering
 \includegraphics{figures/sm_plant_hmat_components}
 \caption{Components used to estimate competition functions shown in Figs.
 {\figPlantHmat} for \plant\ model and the trait height at maturation.
  Panels show: (a) the fitness of a rare invader growing without competition,
  (b) the equilibrium population density of invader when growing in monoculture,
  (c1, c2) the fitness of a rare invader growing in competition with an established
  resident, whose location is indicated by dashed line, and (d1,d2) derived competition
  functions for communities in panels c1 \& c2. Parameters as in Fig.
  {\figPlantHmat}.}
  \label{fig:sm_plant_hmat_components}
\end{figure}

\begin{figure}[h]
 \centering
 \includegraphics{figures/sm_plant_lma_components}
 \caption{Components used to estimate competition functions shown in Figs.
 {\figPlantLma} for \plant\ model and the trait leaf mass per area.
  Panels show: (a) the fitness of a rare invader growing without competition,
  (b) the equilibrium population density of invader when growing in monoculture,
  (c1, c2) the fitness of a rare invader growing in competition with an established
  resident, whose location is indicated by dashed line, and (d1,d2) derived competition
  functions for communities in panels  c1 \& c2. Parameters as in Fig.
  {\figPlantLma}.}
  \label{fig:sm_plant_lma_components}
\end{figure}

\begin{figure}[h]
  \centering
  \includegraphics{figures/sm_rstar_density1}
  \caption{Density dependence in \Rstar: symmetric model}
  \label{fig:rstar_density_dependence}
\end{figure}

\begin{figure}[h]
  \centering
  \includegraphics{figures/sm_rstar_density2}
  \caption{Density dependence in \Rstar: asymmetric model}
  \label{fig:rstar_density_dependence}
\end{figure}

% \clearpage
% \bibliographystyle{amnat}
% \bibliography{references}

\end{document}

%%% Local Variables:
%%% mode: latex
%%% TeX-master: t
%%% TeX-PDF-mode: t
%%% End:
