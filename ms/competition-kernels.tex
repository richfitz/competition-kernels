\documentclass[a4paper,11pt]{article}
\usepackage[osf]{mathpazo}
\usepackage{ms}
\usepackage{natbib}
\usepackage{graphicx}
\usepackage{caption}
\usepackage[labelfont=bf]{caption} % make label for figure bold

% Allow referencing into the supporting information, once that exists.
\IfFileExists{./competition-kernels-supporting.tex}{%
  \usepackage{xr}%
  \externaldocument{competition-kernels-supporting}}{}

\title{competition kernels: shapes and consequences}
\author{}
\date{}
\affiliation{}
\runninghead{}
\keywords{}

\newcommand{\ud}{\ensuremath{\mathrm{d}}}
\newcommand{\sign}{\mathop{\mathrm{sign}}\nolimits}
\newcommand{\Rstar}{\ensuremath{R^*}}

\begin{document}

% Disable words breaking over lines for final submission:
% \raggedright
% \pagestyle{empty}

% \mstitlepage
\parindent=1.5em
\addtolength{\parskip}{.3em}

% \begin{abstract}
% \end{abstract}

\section{Introduction}


%MW version starting here
Central to community ecology has been the metaphor of species packing. The community is seen as fitted into a volume of niche space. Each species that is a potential community member has a fundamental niche somewhere in the volume. The fundamental niche is compressed by competition into a realised niche, as species are packed in alongside each other (Figure \ref{fig:competition-kernels} for one-dimensional version found in [almost all] textbooks). Roots of the species packing metaphor trace to the competitive exclusion principle (Lotka 1925, Volterra 1926, Gause 1934) whereby if species have identical niches or compete for a single resource, all but one are competitively excluded. Hutchinson (1957) added the n-dimensional hypervolume. The  competitive exclusion principle was converted [inverted? made more quantitative?] into the question how dissimilar two species need to be in order to coexist. This question included "limiting similarity" (MacArthur and Levins 1967) \citep{MacArthur-1967}, and also "character displacement" (Brown and Wilson 1956), the evolution of wider trait differences between species where they coexist. By the 1970s, a new generation of textbooks (Whittaker 1970, Krebs 1972, Ricklefs 1973, Pianka 1974) built their account of community ecology around three alternative ways of getting more species into a community: larger total niche space, more overlap between adjacent niches, or narrower niches for each species [could potentially give page or diagram references here?]. Often, these alternatives were discussed as hypotheses for why there are more species in the tropics compared to the temperate zone. 

%MW Haven't yet said anything about the empirical literature looking for overdispersion of traits in communities -- is it important to our Intro, and if so where should it go? Maybe leave till Discussion?

In 1960s theory and 1970s textbooks, niches were typically thought of as utilization functions along a continuous resource axis. The resource in question would be depletable or pre-emptable, in order for overlap of utilization functions to translate into competition. Pairwise competition coefficients (such as alphas in Lotka-Volterra competition) could be calculated from overlap of utilizations functions. If a trait was used as a niche index, it was chosen on grounds of mapping tidily on to a depletable resource, such as beak size for seed-eating finches. Beginning in the 1980s, two different theory strands represented traits more explicitly. One was theory for character (= trait) displacement \citep[e.g.,][]{Taper-1985, Case-2000,   Goldberg-2006}. This flowed on into game theoretical and eventually adaptive-dynamic modeling of trait divergence (Brown 1987, Abrams et al 1993, Geritz et al 1998, 1999, Dieckmann 1999). The other was Tilman's (1982) r* model, where the "trait" was the level to which a species could deplete a resource while still sustaining its population. The resource was homogeneous, such as dissolved nitrate in a water body, and this was a distinct break from the formulation of a resource with such as seeds with a continuously varying property such as seed size. Tilman's (1982) model would have a single winner on a single resource (the species with lowest r*), but given two or more resources multiple species could coexist on different ratios between resources. 

%MW possibly better to leave r* and the nature of coexistence made possible till it is introduced for analysis? 

Competition kernels are best thought of within the adaptive-dynamics tradition of fitness landscapes. Species exist, coexist and evolve on a landscape where the axes are evolvable traits. A competition kernel represents the impact of a species on this landscape -- the drawdown of fitness across the landscape as it relates to the trait-values both of the species exerting a competitive effect and of the species experiencing the effect. Competition kernels are absolutely central to the question how species interact in competitive communities. And yet, surprisingly little is known about alternative possible shapes for competition kernels and their consequences for community assembly. 

By far the most common type of utilization curve or competition kernel in textbooks and in theoretical literature is that they are unimodal, decreasing with trait-distance away from the species that is exerting the competitive effect. For example Birand (2014) wrote "we assume that the competition function is Gaussian, which is typically assumed in much of the literature (e.g., Dieckmann and Doebeli 1999; Bolnick 2006)". 

%MW If we wanted to abbreviate the history, we could maybe at this point go straight to the theory that shows you don't necessarily get species-spacing from gaussian kernels

The species packing metaphor has persisted up to the present day as the mainstream approach to community ecology. (Indeed,  Krebs (6th edn 2009), Ricklefs and Miller (4th edn 2000) and Pianka (6th edition 2000) continue also.) Yet in parallel, three other strands of thinking had already been established by the 1970s. 

Terrestrial plant ecology had always found it difficult to accept the idea that different species can only coexist by using different resources, since plant species clearly all use the same sources of light, soil water and mineral nutrients as each other, with only limited opportunities for differentiation by soil depth and by chemical forms of nutrients. One possibility arises from species subdividing a successional-time niche axis in landscapes envisaged as disturbance mosaics (Connell 1978). If plant species make most contribution to their population increase in different years following disturbance, then they certainly are depleting different light and water from each other, and quite possibly different soil nutrients \citep{Huston-1987, Kohyama-1993}. 
%[MW -- these refs imported from Rich's notes -- need to check whether they're appropriate for this point]. 

In a more radical break from subdividing continuous niche axes, models arose that visualised communities as a chessboard of living sites, each able to be occupied by one sessile individual (Yodzis 1979). Coexistence could arise by colonization-competition trade-offs (Levins 1969, Levins and Culver 1971). Where competition is size-asymmetric as in plants competing for light (Harper 1977, Weiner 1990), or where there is advantage to individuals established in territories, as in some coral reef fish, then chance effects in the occupation of vacant living sites can slow competitive exclusion by "lottery" (Sale 1977) or "storage effect" (Chesson and Warner 1981) processes. There have been several theory-lineages within this "sessile dynamics framework" (Fagerstrom and Westoby 1996), and "neutral theory" (Hubbell 2001) also envisions the world as a (very large) chessboard of living sites on which stochastic processes occur.   

Another strand of thinking parallel to the species packing metaphor has been ecological strategy theory (e.g. Grime 1974, 1977, 1979, 1988, Southwood 1977, Greenslade 1983). In the best-developed 'CSR' version of this, species were arrayed on a triangle with competition-winning, stress-tolerating and ruderal strategies at the corners. Stress-tolerating included persistence under competition. Predictions arose about the strategy-mixture expected in different physical settings and at different stages of succession (Grime 1979, 2001). Then during the 1990s the suggestion was made to sidestep continuing disputes about the definition of competitiveness by focusing instead on measurable traits and using them as strategy axes (Westoby 1998, Weiher et al 1999, Westoby et al 2002). This suggestion made possible studies of coordination and trade-offs between traits on a worldwide scale (e.g. Wright et al 2004, Moles et al 2005, Zanne et al 2010, Cornwell et al 2014). However, this "trait ecology" has yet to return successfully to the question how species fit together in communities. By being registered on trait axes, competition kernels have the potential to link this world-scale empirical trait ecology back to questions of species assembly into communities. 

A third strand of thinking, already established during the 1970s in parallel with the species packing metaphor, was theory showing that with gaussian kernels along a single niche axis, there is in fact no limit to similarity (May 1972, Abrams 1975, 1983). Competitive exclusion can become slow because two competitors are similar in their resource use as well as because they are different. "Continuous coexistence" is a theoretically possible outcome, with resources occupied by many species spaced from each other in resource use by very small amounts. Although this strand of theory might have been expected to undermine the mainstream approach to community ecology via the species packing metaphor, in actuality it did not have the effect of dislodging species packing from standard textbooks, probably because it did not really offer an alternative program for connecting theory to field measurement.  After 2000, this line of theory began to explore more systematically the consequences of possible kernel-shapes for community assembly (Schaffer and van Nes 2006, other literature, Leimar et al 2013).   

In summary, the current situation is that it is clear competition kernels on fitness landscapes must be the key to understanding community assembly, and clear also that symmetric unimodal or gaussian kernels are realistic for only a minority of real-world communities. At the same time, the menagerie of possible competition kernels has only begun to be catalogued. Much less have the implications of different kernels for community assembly been worked through. The literature up to the present has mainly been focused on gaussian kernels, perhaps because they were mistakenly thought to be prevalent, or out of fascination with the "continuous coexistence" they can produce. Surprisingly little attention has been paid to the question what shapes are actually likely for competition kernels, given known mechanisms for resource depletion in relation to traits. We argue that clarifying plausible shapes for competition kernels and their consequences will be an essential foundation before a robust community ecology can be constructed.  

In this paper we investigate competition kernels that are produced in selected models of growth and reproduction where competition and the mechanics of the resource being depleted are represented mechanistically. In other words, competition kernels are emergent properties from these models, not assumptions adopted into them. In so doing, we
\begin{itemize} 
\item 
\item introduce standard methods for characterizing competition kernels for any selected model
\item ask what is the shape of the competition function emerging from the model -- Is it Gaussian? Is it symmetric?
\item ask whether the shape of the competition function varies with traits of the resident species
\item ask whether the shape of the competition function varies with density of the resident species
\end{itemize}

\textit{This work is undertaken in the belief that if competition kernels with respect to traits were better understood, this could open a constructive path to}
\begin{itemize}
\item Measure real competition kernels in nature. (Measuring competition as a function of traits rather than between a pair of species may often be difficult in the field, but knowledge of plausible kernel shapes certainly should help.)
\item Predict trait-mixtures. This is arguably intrinsically more powerful than predicting species richness. Moreover, it leads on to the question whether a signature of competition is expected in extant trait-mixtures. If we knew what type of trait-mixture signature should be expected for different mechanisms of competition, that could become a powerful tool for relating theory to real-world communities. 
\end{itemize}  




%MW bits left over from Rich's notes  
Despite their centrality in modelling and ecological thought, models
of competition remain largely unchanged from their original
formulation in the 1960's, with the most popular version due to
\citet{MacArthur-1967}.
% Probably point at the methods here.
Generally species are assumed to interact following the Lotka-Volterra
equations (REFS), with the intensity of competition between two
species ($\alpha_{ij}$) being some function of the two species traits
(for a trait axis $x$, $\alpha_{ij} = \alpha(x_i, x_j)$; see Methods).
%
This formulation implies among other things that competition is linear
in intensity with species abundances, is additive across multiple
competing species, and that the \emph{shape} of competition is an
intrinsic (i.e., not context-dependent) property of pairs of species.
%
While possibly justifiable as low-order approximations, these
assumptions have been widely criticised as long as the models have
been used
\citep[e.g.][]{Andrewartha-1953,May-1972,Abrams-1975}.
%MW question: do we want to cite Andrewartha? -- what is his criticism exactly?
% TODO: I'd love something here where we can report on any attempts to
% measure the *shape* (not intensity) of competition in nature.
%
%MW well there's Georges's work I suppose
The shape of the competition kernel with respect to resources and
species traits has received less attention \citep[but
see][]{Abrams-2008,Leimar-2013}; typically a Gaussian or a form of
$\alpha(x_i, x_j) \propto \exp((x_i - x_j)^4)$ is used, or more rarely
an asymmetric logistic.
%




\section{Models}

% In the first paragraph, I guess we describe the entire approach as
% well as possible.

We investigated the dynamics of two quite different models -- TREE and R-star
-- where the competitive impact of one species on another arises via the
effect of individuals on a shared resource. In each model, species traits
influence the dynamics of resource use and thus competitive interactions. For
each model, we compute the core quantities used in the Lotka-Volterra
equations (growth rate, carrying capacity,  per-capita rate of population
increase) and from that infer the shapes of competition functions.

The standard Lotka-Volterra equations express the per capita rate of increase
in the abundance $N_i$ of species-$i$as a product of two terms. The first
gives the maximum possible per-capita growth rate $r_i$. The second expresses
the degree to which that maximum is realised, and thus includes any reductions
due to competition with other species. A commonly used form for Lotka-Volterra dynamics
is
\begin{equation}
\label{eq:LV}
\frac{\ud N_i}{\ud t} \frac{1}{N_i} =  r_i \left(1 - \sum_j
                                      \frac{\alpha_{ij} N_j}{K_i}\right),
\end{equation}
where $K_i$ is the equilibrium value for $N_i$ when species-$i$ is grown in a
monoculture  (i.e. carrying capacity), and $\alpha_{ij}$ is the per-capita
competitive effect of species-$j$ on species-$i$. By convention, competition
functions are expressed as the effect of species-$j$ on species-$i$, relative
to the effect of species-$i$ on itself. Thus $\alpha_{ii}\equiv 1$. It is
also worth noting, that there are other forms for eq. \ref{eq:LV}. The alternative
forms differ subtly in the nature of density dependence, but retain the same
overall structure of \ref{eq:LV}, wherein a maximum possible growth rate is multiplied
by a second term incorporating reductions due to competition. For simplicity,
we focus on a single version of the Lotka-Volterra dynamics.

We are interested in situations where the different parameters in eq.
\ref{eq:LV} are functions  of traits. Moreover, we restrict our
attention to situations where a single type with traits
$x_I$ is trying to invade a community comprising a single resident type with
traits $x_R$.  The fitness $f(x_I, x_R)$ of the invading type is
then
\begin{equation}
  \label{eq:LVi}
   f(x_I, x_R) = \frac{\ud N_i}{\ud t} \frac{1}{N_i} =
    r(x_I)\left(1 - \alpha(x_I, x_R) \frac{K(x_R)}{K(x_I)}\right).
\end{equation}
The reason for this narrowing of focus is that it is not possible to
infer the shapes of competition functions in situations with more than one
resident, because the number of unknowns increases faster than the number of
species (refs?).

To estimate the shapes of competition functions embedded within the R-star
and TREE models, we use those models to calculate all of the trait-dependent
terms in eq. \ref{eq:LVi} except $\alpha(x_I, x_R)$:
\begin{enumerate}
  \item $f(x_I, x_R)$: fitness of rare invader with traits $x_I$  in competition
  with a established resident with traits $x_R$
  \item $r(x_I)$: maximum possible fitness of invader, i.e. in absence of competition,
  \item $K(x_I)$: equilibrium population density of invader when growing in monoculture.
  \item $K(x_R)$: equilibrium population density of resident when growing in
  monoculture.
\end{enumerate}
We then solved for $\alpha(x_I, x_R)$ by rearranging  \ref{eq:LVi} to give:
\begin{equation}
  \label{eq:alpha}
 \alpha(x_I, x_R) = \frac{K(x_I)}{K(x_R)}
  \left(1 - \frac{f(x_I, x_R)}{r(x_I)}\right).
\end{equation}

\subsection{R-star model}

% \item As originally formulated, \Rstar\ does not include traits
% \item We adapt a formulation (due to whom?) where a ``trait'' is a
%   measure of specialisation/generalisation.

In the first model (R-star) individuals compete for two essential and
depletable resources. This model is one of the best studied and cited models
of resource competition \citep{Tilman-1977, Tilman-1982, Huisman-1999},
originating from a study on exploitative competition for any pair of generic
resources \citep{Leon-1975}. The model is unstructured, with the per capita
growth rates for each species simply depending on the availability of two
resources. We assume both resources are necessary (i.e. non-substitutable), so
the growth rate of each species is determined by the resource which is
currently most limiting. We implemented a version where the specific growth
rates follow a Monod equation \citet{Huisman-1999}, but which is otherwise
similar to the widely-cited implementation by Tilman \citep{Tilman-1977,
Tilman-1982}.  The resource availabilities, in turn, depend on the rates of
resource supply and consumption. Individuals compete with one another by
drawing down the joint resource pools. Such a system is easily described via a
minimal set of equations (Appendix \ref{Rstar}).


Following previous authors \citep{Tilman-1985, Schreiber-2003, Fox-2008} we
focus on a trait determining the relative amount of the two resources
required for growth, i.e. the stoichiometry of resource needs. Species can
vary from requiring mostly resource 1, to equals amounts of both resources, to
mostly resource 2. This dependence is achieved by making uptake saturation
constants (see appendix  \ref{Rstar} for details) for resource 1 inversely related
to the requirement of resource 2:
\begin{equation}
\label{eq:R3}
U_{1i} = 1-U_{2i}.
\end{equation}

\subsection{TREE model}

In the second model (TREE) individual plants compete for light within a patch
and size-structured metapopulation \citep{Falster-2011, Falster-2015}.
Vegetation in this model comprises a large number of patches subject to a
prevailing disturbance. Disturbances remove established vegetation within a
patch, which is then recolonised via seed dispersal from other patches.
Individual plants progress through their life-cycle, from seeds until mature
adults, with growth survival and fecundity rates determined by the combined
effect of an individual's traits, its current size and the degree of shading
from other plants within the same patch. The amount of shading at a given
height is computed from the cumulative leaf area above that height; short-term
competitive effects are therefore size-asymmetric. The sole effect of shading
is to alter an individual's photosynthetic income. Following
\citep{Falster-2015}, the fitness of individuals with a given type is
calculated by integrating seed output across the entire metapopulation.

We use the TREE model to estimate trait-based competitive functions for
two traits whose evolutionary dynamics have recently been investigated
\citep{Falster-2015}: leaf mass per area (LMA) and height at maturation
(HMAT). In the model, LMA moderates a physiological trade-off between the cost
of building an extra unit of leaf area and the rate of leaf turnover.
Variation in LMA thereby underpins a trade-off between height growth rate and
shade-tolerance of seedlings. Meanwhile, HMAT moderates the allocation of
surplus energy between growth and seed production, as an individual increases in
size.

\clearpage

\section{Results}

Main results:

\begin{itemize}
\item We saw evidence of asymmetric kernels from height competition,
  as expected and as exist in the literature
\item We saw kernels that didn't look remotely Gaussian
\item Carrying capacities were more Gaussian for TREE, I think.
\end{itemize}

Other things to look into

\begin{itemize}
\item Show that the bumps in the LMA landscape facilitate coexistance,
  and see if we can remap the trait/resource axis onto time somehow?
\end{itemize}

\subsection{Figures}


\begin{enumerate}
\item R-star competition estimation
  \begin{itemize}
  \item general shape
  \item Change in shape with parameters and/or resident position --
    residents probably more important though as parameters should
    change kernels extensively in most modls.
  \item Not many panels for each figure to keep it simple.  But make
    figures like 2 for the SM that show the different ingredients that
    are going into the calculation.
  \item Show the different measures of compettion provided that they
    all fit on the page nicely; \citet{Abrams-2008} makes the point
    that if different measures don't line up then competition is
    density dependent.
  \end{itemize}
\item Hmat competition estimation
  \begin{itemize}
  \item Choose a case where we don't have coexistence to keep it
    separate.
  \item First because it most closely matches \citet{Kisdi-1999} and
    because the results are really simple and don't look too weird.
  \end{itemize}
\item LMA competition estimation
  \begin{itemize}
  \item For LMA we want to get the curve into a bifurcation like point
    where coexistence is possible on both sides
  \item Show that the points that coexistence is possible on the RHS
    are the dips in the competition function to make the case that
    this is a fundamentally different way that competition might allow
    coexistence in models with non-linear density dependent
    competition.
  \end{itemize}
\item Density dependence in competition models
  \begin{itemize}
  \item Something showing that the competition kernels are density
    dependent.  Because this is a three dimensional functions (trait,
    competition strength and resident density) we can only show slices
    through two of these.  At different densities show how the
    kernels look with respect to traits.  Possibly do that at two
    different resident communities.
  \item Need to pick a model to do this with - could be any of the
    three and for completeness we might want to add them to the SM.
  \end{itemize}
\end{enumerate}

\clearpage
\section{Discussion}

\textit{Main issues of competition models illustrated by our analysis}
\begin{itemize}
\item No clear generalisation yet about competition kernel shapes.
  \begin{itemize}
  \item our LMA kernel looks a little like the
    \citet{Roughgarden-1979} offset Gaussian, but the ridges are
    likely important for later species (we need to actually show this
    however)
  \item our HMAT kernel looks like the \citet{Kisdi-1999} asymmetric
    kernel, and the dynamics are fairly similar (needs
    demonstrating).
  \item The R* kernels just look weird, but that's because the
    \emph{trait} doesn't really map onto the resource very directly;
    the trait is the preference for two different resources, which
    can't map onto any single resource very well.
  \end{itemize}
\item Kernels other than Gaussian may lead to coexistence more easily
  than Gaussian.
  \begin{itemize}
  \item It's not all bad news; kernels like the LMA kernel look
    ridiculous and you're not going to be allowed to propose them for
    a modelling paper.
  \item But they arise from size-structured dynamics in the model, so
    may not be that uncommon (worth noting that almost all the models
    are really of discrete exchangeable adults and that size/age
    structure was flagged as an issue ages ago but I've lost the
    paper).
  \end{itemize}
\item Density dependent competition emerged in every model, as
  predicted by \citet{Abrams-1980}, who argued that not satisfying
  implicit assumptions would lead to widespread non-constant density
  dependence.
\item More than density dependence, the \emph{shape} of the kernel
  changes with density, moving in position as the resident population
  density changes.  Kernels of this type simply haven't been studied
  and it's not actually very clear if kernels of this type might
  stretch the concept beyond its useful limit.
\item Outcome of competition is governed by the relative shapes of r,
  K and $\alpha$ but most modelling focusses only on the competition
  kernel, setting $K$ to be uniform or Gaussian and setting $r$ to be
  uniform.  The exception is asymmetric competition studies that tend
  to use asymmetric $r$ functions (pointing in the opposite direction
  of competitive ability). We don't know what \emph{any} of these look
  like in nature.
\end{itemize}

\textit{Traits are a productive way forward -- for what?}
\begin{itemize}
\item What did we want to say here, really?  Traits are already how
  people are thinking about kernels, but the work on resources hasn't
  caught up (see last point)
\item In a way, we're arguing to turn the way that we've looked at
  competition around: rather than asking ``given we know species
  coexist, what do we have to do to competition kernels'', we're
  saying: ``given we have data on species traits, can we just look at
  the shapes that competition functions actually take and use that to
  guide future modelling''.
\end{itemize}

\textit{(Maybe) link to field measurement of competition.}
\begin{itemize}
\item Most manipulative measures of competition in the field are going
  to give \emph{discrete} estimates of $\alpha$: the amount of
  competition exerted by or felt by a single species in the context of
  some community.
\item Because original, \citet{MacArthur-1967}, version of competition
  kernels was designed around resource utilisation functions, it
  should be possible to estimate competition functions and kernels
  from shapes of those functions if they were estimatable.
\item Does our way offer any better way forward?  The non-manipulative
  work that Georges is doing with the working group data, or that
  James is looking at with the BCI data might help; we have large data
  sets of traits and growth; the signature of competition may show up
  there (perhaps we could put it on a manifold\ldots).
\end{itemize}

\textit{Trait-resource utility linkage is important, but largely
  ignored.  Mechanistic models provide some sort of way forward.}
\begin{itemize}
\item Of course we don't actually make this link in the model --
  perhaps we should
\item The 1:1 mapping happened very quietly, but the resource axis has
  gradually been entirely replaced by trait axes.  We argue that
  neither are likely to exhibit behaviour like the
  \citet{MacArthur-1967} model in natural systems.
\item There is a shift in the 1980s from modelling limiting
  similarity in terms of resources to modelling trait
  divergence/convergence and disruptive selection.
  \begin{itemize}
  \item Begins with \citet{Taper-1985} and \citet{Brown-1987-140}; these
    make explicit a direct 1:1 trait:resource mapping implied by the
    previous models.
  \item Continued as the models were used in adaptive dynamics
    \citep[e.g.,][]{Dieckmann-1999} and then picked up by physicists
    \citep[e.g.,][]{Leimar-2013}
  \item For vertebrate animals, this connection has always been fairly
    defensible for the sorts of traits people think about (e.g., finch
    beak size --- Lack), but for plants the resource axis is hard: most
    plants would rather more light.  Shifting to traits with some
    complicated non-linear map between traits and resources offers a way
    forward, though it makes the links to resources and underlying
    process muddier.
  \end{itemize}
\item It's hard to bridge between abstract models of competition that
  inform most theory with the sort of traits and species data that we
  have for plant communities.
\end{itemize}

% Other ported notes:
% \item The \Rstar\ model is formulated with the same ingredients as LV
%   but gives totally different results.  This is a concern if the most
%   commonly used models of explicit and implicit competition do not
% \item Kernels such as the LMA one may promote coexistence in ways that
%   are totally different from the models based on the Gaussian
%   kernels.  As such, thinking narrowly about competition facilitating
%   coexistance in this way may be misleading.
% \item The arguments about limiting similarity are probably very
%   sensitive to the assumptions about constant per-capita rate of
%   competition.  If competition increases or decreases with the density
%   of species, this should affect how closesly they can pack.  These
%   ideas have previously been explored by Abrams I believe.
% \item If we let $r$ and $K$ vary with species traits and let
%   $\alpha(x, y)$ vary with traits and densities in an arbitrary way,
%   do we hae anything tractable?  Is this just saying ``it's all
%   hopeless''?  Are there simple LV-style models we could possibly come
%   up with that mimic complex models or realistic situations in ways
%   that make any sense?
%   agree.
% \item This is not meant to be a criticism of simple models \textit{per
%     se}; simple models are essential to ecology, and there are many
%   resons to model competition and anything much more complicated than
%   Gaussian kernels quickly becomes intractable.  More concerning is
%   that we don't have a strong idea about how departures from the
%   Gaussian kernel will impact models of coexistance or of other models
%   for which competition is a key ingredient.  There is some quite
%   theoretical work on this (e.g. some recent Doebeli work), but it's
%   still a long way removed from the weird shapes here.
% \item Ideally we can odentify features that mechanistic models, models
%   with implicit competition, and emprical data have in common and make
%   sure that our models that invoke explicit competition are robust to
%   these features.
% \item Can measuring competition in mechanistic models help
%   bridge theory and data? May help interpret data fromnonmanipulative
%   estimation of competition intensity (e.g., Georges' work).

\textit{Conclusions}
\begin{itemize}
\item Coming up with a positive note here might be hard.
\end{itemize}

\section{Figures}

\begin{figure}[h]
  \centering
  % \includegraphics{figs/competition-kernels}
  \caption{\textbf{Classic text-book view of niche differentiation.
  Along the x-axis is a resource to be partitioned, such as food or prey size.} Lines
  represent hypothetical resource utilisation curves for five different species,
  indicating the propernsity of that species to use resources of a given size.
  The inetnsity of competition between two species is indicated by the degree
  of overlap in their resource utilisation curves. \citet{MacArthur-1967} proposed
  that competition would limit how closely species could sit relative to one
  another, via the width of the ultisation curve. Soon after it's publication,
  this idea of `limiting similairty` appeared in ecology textbooks, with similar
  figures to the one shown appearing in the earliest and also the most recent
  editions of leading textbooks \citep{begon-1986, begon-2006, Krebs-1978, Krebs-2013,
  Ricklefs-1973, Ricklefs-1999}.}
  % begon-1986: Fig 7.9, pg 263
  % begon-2006: Fig 8.29, pg 257
  % Krebs-1978: Fig 12.2, pg 233
  % Krebs-2013: ??
  % Ricklefs-1973: Fig. 36.13, pg 526
  % Ricklefs-1999: Fig 29.19, pg 602
  \label{fig:competition-kernels}
\end{figure}

\begin{figure}[h]
  \centering
 \caption{\textbf{Method for estimating the shape the competition function embedded
  in a mechansitic resource-based model.} To illustrate the method, we present 
  results from a model with known solution \citep{Dieckmann-1999} (see Appendix
  \ref{DD99} for details). For each
  model we estimate: (a) the fitness of a rare invader growing without competition,
  (b) the equilibrium population density of invader when growing in monoculture,
  (c1, c2) the fitness of a rare invader growing in competition with an established
  resident. From these elsements, we can estimate the shape of the competition
  function (d1,d2). Panels a-b do not vary with the resident community whereas
  panels c-d may.  As intended the method recovers the Guassian-shaped competition
  functions assumed by this toy model. Moreover, comparison of panels d1 \& d2
  shows that the shape of the function does not vary with resdient trait.}   
  \label{fig:components}
\end{figure}


\clearpage

\bibliographystyle{amnat}
\bibliography{references}

\section{Appendices}

\subsection{Details about generic Lotka-Volterra model with 
Gaussian competition}\label{DD99}

\citet{Dieckmann-1999} analyse a system following the Lotka-Volterra dynamics
described in equation eq. \ref{eq:LVi}, where maximum growth rate is constant
across species, carrying capacity is a Gaussian
function of the invader trait, and the competition function is a Gaussian 
based on the difference between traits of the invader and resident:
\begin{subequations}
\begin{align}
\label{eq:DD}
r(x_I) &= r_0, \\
K(x_I) &= K_0 \exp\left(-\frac{\left(x_I-x_0\right)^2}{2\sigma^2_K}\right),\\
\alpha(x_I, x_R) &= \exp\left(-\frac{\left(x_I-x_R\right)^2}{2\sigma^2_C}\right).
\end{align}
\end{subequations}

For the analyses presented in Figure \ref{fig:components} we let $r_0=1.0$, $K_0=500$,
 $s2_C=1$, and $s2_C=0.16$.

\subsection{Details about R-star model}\label{Rstar}

We consider an unstructured system where individuals compete for two
resources. Let $N_i$ be the abundance of species-$i$ and $R_1, R_2$ be the
abundances of the two resources. The population of species-$i$ increases with
per-capita
growth rate $\mu_i(R_1,R_2)$, a function of the resource availabilities, and
decreases with per-capita mortality rate $m$. Species-$i$ depletes resource-$k$
at rate $c_{ki} \,  N_i \, \mu_i(R_1,R_2)$, where $k = 1, 2$ and $c_{ki}$ is
the content of resource-$k$ in species-$i$. Resources
are replenished
with supply rate $S_k$ and decay at rate $D_k$. Under these assumptions, the
dynamics of species and resource abundance are given by
\begin{subequations}
  \begin{align}
\label{eq:R1}
\frac{dN_i}{dt} &= N_i \, \mu_i(R_1,R_2) - N_i \, m \\
\frac{dR_k}{dt} &= D \, (S_k- R_k) - \sum_{i=1}^{n}{c_{ki} \, N_i \,
  \mu_i(R_1, R_2)},
\end{align}
\end{subequations}
where $n$ is the number of species and $k=1,2$ is the number of resources.

The birth rate $\mu$ for each species are assumed to follow a Monod equation,
and be determined by the resource which is most limiting, such that
\begin{equation}
\label{eq:R2}
\mu_i(R_1, R_2) = r_i \, \min \left(\frac{R_1}{U_{1i} + R_1},
                                    \frac{R_2}{U_{2i} + R_2} \right).
\end{equation}
% NOTE: In above equation `U` was previously labelled `K`. I suggest we use U
% over K because K is already used in the Lotka Volterra equations, so reusing
% same symbol here is likely to cause confusion.
In eq. \ref{eq:R2}, $r_i$ is the maximal growth rate rate of species $i$, $U_
{ki}$ is the half saturation constant for species-$i$limited by resource $k$.

\subsection{Details about TREE model}\label{TREE}

Excerpt from  TREE model. Hold off writing this until know whether paper describing
the model has been published or not.

\end{document}

%%% Local Variables:
%%% mode: latex
%%% TeX-master: t
%%% TeX-PDF-mode: t
%%% End:
