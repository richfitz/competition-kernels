\documentclass[a4paper,11pt]{article}
\usepackage[osf]{mathpazo}
\usepackage{ms}
\usepackage{natbib}
\usepackage{graphicx}
\usepackage{caption}

% Allow referencing into the supporting information, once that exists.
\IfFileExists{./competition-kernels-supporting.tex}{%
  \usepackage{xr}%
  \externaldocument{competition-kernels-supporting}}{}

\title{Something about competition kernels}
\author{}
\date{}
\affiliation{}
\runninghead{}
\keywords{}

\newcommand{\ud}{\ensuremath{\mathrm{d}}}
\newcommand{\sign}{\mathop{\mathrm{sign}}\nolimits}

\begin{document}

% Disable words breaking over lines for final submission:
% \raggedright
% \pagestyle{empty}

% \mstitlepage
\parindent=1.5em
\addtolength{\parskip}{.3em}

% \begin{abstract}
% \end{abstract}

\section{Introduction}

\textit{Competition kernels are widely used in both models and arguments}
\begin{itemize}
\item A fundamental measure of community dynamics: kernels represent
  the fitness landscape drawdown
\item example usage:
  \begin{itemize}
  \item limiting similarity \citet{MacArthur-1967}
  \item coexistence
  \item character displacement \citep[e.g.,][]{Taper-1985, Case-2000,
      Goldberg-2006}
  \item game theory \citep[e.g.,][]{Brown-1987, Brown-1987-140}
  \item disruptive selection and adaptive dynamics \citep{Dieckmann-1999}
  \end{itemize}
\item Species packing remains a hugely influential concept: the
  niche is usually reprensented as a Gaussian disstribution in one
  dimension (e.g., Ricklefs' textbooks over the years: p 602 in
  1999, v4).
\end{itemize}

\textit{Competition kernels, and models of competition, are largely
  unchanged from their original formulation in the 1960s.}
\begin{itemize}
\item Based on Lotka-Volterra equations (\citealt{MacArthur-1967}
  references Volterra 1926), with a function for competition in terms
  of a pair of species traits; the main fomulation due to
  \citet{MacArthur-1967}
\item These models have been criticised from the outset:
\begin{itemize}
\item \citet{Andrewartha-1953}
\item \citet{May-1972}
\item \citet{Neill-1974}: competition functions not meaningful
  properties of species pairs, are density and context dependent.
\item \citet{Abrams-1975}: sensitive to details about how resources
  are used.
\item \citet{Abrams-1980}: implicit assumptions likely to be violated
  leading to density dependent competition
\end{itemize}
\item Surprisingly, most of the criticism of shape focusses on the
  ease with which \emph{continuous coexistence} is possible, rather
  than focussing on what shapes are meaningful: we have very little
  information on the likely shapes in nature aside from the assertion
  that a Gaussian is a likely reasonable low-order approximation.
\end{itemize}

\textit{There is a shift in the 1980s from modelling limiting
  similarity in terms of resources to modelling trait
  divergence/convergence and disruptive selection.}
\begin{itemize}
\item Begins with \citet{Taper-1985} and \citet{Brown-1987-140}; these
  make explicit a direct 1:1 trait:resource mapping implied by the
  previous models.
\item Continued as the models were used in adaptive dynamics
  \citep[e.g.,][]{Dieckmann-1999} and then picked up by physicists
  \citep[e.g.,][]{Leimar-2013}
\item For vertebrate animals, this connection has always been fairly
  defendable for the sorts of traits people think about (e.g., finch
  beak size --- Lack), but for plants the resource axis is hard: most
  plants would rather more light.  Shifting to traits with some
  complicated nonlinear map between traits and resources offers a way
  forward, though it makes the links to resources and underlying
  process muddier.
\end{itemize}

\textit{At the same time, other strands of modelling aimed to link
  species traits, competition and coexistence via more explicit
  representation of resoucres.}
\begin{itemize}
\item Soil resources, via R* \citep{Tilman-1980,Tilman-1982}. this line of theory
  aimed to link  resource
  dynamics to competition via species competing for different pools of
  resources.  It contains many of the same ingredients as the
  \citet{MacArthur-1967} model, but with slower replenishment of
  resources.
\item Space, via Sessile dynamics, lottery models, competition-colonisation
  models, through to neutral theory. All assumed total number of fixed sites,
  with individuals competing for these.
\item Light, via models of successional turnover \citep{Huston-1987, Kohyama-1993}.
  Focus on size-structure and assymetric nature of competition.
\item Important point in all of these is that compeitive affect of one species
  on another operates via competition for resource. This makes the shape of
  competiton coefficents are an emergent  property of system.
\end{itemize}

\textit{If we understood competition kernels with respect to traits,
  we could}
\begin{itemize}
\item Measure real competition kernels in nature
\item Predict trait-mixtures better than species mixtures.  Is there a
  signature of competition in extant trait-mixtures?  What would it
  be?
\item Measuring competition along trait gradients in the field likely
  intractable, but possible in models where competition emerges from
  first principle ingredients about biological processes that we think
  are important.
\item Basing an entire field on competition kernels where we don't
  even know their shape seems like a house of cards.
\end{itemize}

\textit{Aim of this paper}: to investigate the shape and behaviour of
competition kernels in mechanstic models of growth and reproduction.
\begin{itemize}
\item In a sense our approach is similar to \citet{Draghi-2012}; we're
  going to look at models that are simple enough to push around but
  complicated enough that we don't immediately know the emergent
  behaviours and use this as a proxy to do things we can't easily do
  in natural systems.
\end{itemize}

Key questions asked of each model:

\begin{itemize}
\item What is the shape of the compeition function emerging from the model? Is
it gaussian? Is it symmetric?
\item Does the shape of the compeition function vary with traits of the
resident species?
\item Does the shape of the compeition function vary with traits of density
of the resident species?
\end{itemize}


\section{Models}

% In the first paragraph, I guess we describe the entire approach as
% well as possible.

We invetsigated the dynamics of two quite different models -- TREE and RStar
-- where the compeitive imnpact of one species on another arises via the
effect of individuals on a shared resource. In each model, species traits
influence the dynamics of resource use and thus compeititive interactions. For
each model, we compute the core quantities used in the Lotka-Volterra
equations (growth rate, carrying capacity,  per-capita rate of population
increase) and from that infer the shapes of competition functions.

The standard Lotka-Volterra equations express the per capita rate of incraese
in the abundnance $N_i$ of species $i$ as a product of two terms. The first
gives the maximum possible per-capita growth rate $r_i$. The second expresses
the degree to which that maximum is realised, and thus includes any reductions
due to compeiton. A commonly used form for Lotka-Volterra dyanmics is

\begin{equation}
\label{eq:LV}
\frac{\ud N_i}{\ud t} \frac{1}{N_i} =  r_i \left(1 - \sum_j
                                      \frac{\alpha(i,j) N_j}{K_i}\right),
\end{equation}

where $K_i$ is the euqilibirum value for $N_i$ when species-$i$ is grown in a
monoculture  (i.e. carrying capacity), and $\alpha(i,j)$ is the per-capita
compeitive effect of species-$j$ on species-$i$. (There are other forms for eq
\ref{eq:LV} that we do not consider further here. While these differ subtly in
the nature of density dependence, alternative forms retain the same overall
structure of a maximum possible rate multiplied by some reduction due to
competition). By convention, comptitive efefcts are expressed as the effct of
species-$j$ on species-$i$, relative to the effct of species-$i$ on itself.
Thus $\alpha(i, i) \equiv 1$.

We are inetrested in situations where the different parameters in eq.
\ref{eq:LV} are functions  of traits. Moreover, we restrict our
attention to situations where a single mutatnt type invading type with traits
$x_I$ is trying to invade a community of single residnet type with traits
$x_R$. The reason for this narrowing of focus is that it is not possible to
infer the shapes of competition coefficients in  sitautions with more than one
resident, because the number of unknowns increases faster than the number of
species. (?)  The invasion fitness $f(x_I, x_R)$ of the mutant type is then

\begin{equation}
  \label{eq:LVi}
   f(x_I, x_R) =
    r(x_I)\left(1 - \frac{\alpha(x_I, x_R) N(x_R)}{K(x_I)}\right).
\end{equation}

To estimate the shapes of competition coefffiecients
embedded within the Rstar and TREE models, we use those models to calculate all
the tarit-dependent terms in eq. \ref{eq:LVi} except $\alpha(x_I, x_R)$. We can
then solve for $\alpha(x_I, x_R)$ by rearranging  \ref{eq:LVi} to give:

\begin{equation}
  \label{eq:alpha}
 \alpha(x_I, x_R) = \frac{K(x_I)}{N(x_R)}
  \left(1 - \frac{f(x_I, x_R)}{r(x_I)}\right).
\end{equation}

We maniuplated the models
such that the following quantities
could be extracted: i) $N$ --- a meausre of current density, ii) $r(x)$ ---
this is the rate of growth in an empty environment (i.e., mutant fitness
in the empty environment) iii) $K(x)$ --- this is maximum population
density (equilibrium seed rain in a monoculture).


\subsection{Tilman's R-star model}

In the first model (R-star) individuals compete for two soil
resources. The model is one of the best studied model of competition
for resources \citep{Tilman-1982}, which find its roots in \citet{Leon-1975}. We focus on a version of the
model in which the specific growth rates follow the Monod
equation and resources are essential as in \citet{Tilman-1977} or \citet{Huisman-1999}.

\begin{equation}
\label{eq:R1}
\frac{dN_i}{dt} = N_i \, (\mu_i(R_1, \cdots,  R_n) – m) \\
\frac{dR_j}{dt} = D \, (S_j – R_j) - \sum_{i=1}^{n}{c_{ji} \,
  \mu_i(R_1, \cdots,  R_n) \, N_i}
\end{equation}

Where $N_i$ is the population abundance of species $i$, and $R_j$ is the
abundance of essential resource $j$ ($j = 1, 2$). $m$ is the
mortality rate, $D$ is the rate of resources wash out of the system,
$S_j$ is the influent concentration of resource $j$, $c_{ji}$ is the
content of resource $j$ in species $i$ (or stoichiometry), and
$\mu_i(R_1, \cdots,  R_n)$ is the growth rate of species $i$.

The specific growth rates follow the Monod Equation for essential
resources (determined by the resource that is most limiting according
to the ‘law of minimum”)  \citep{Tilman-1977,Huisman-1999}.
\begin{equation}
\label{eq:R2}
\mu_i(R_1, \cdots,  R_n) = min \left( \frac{r R_1}{K_{1i} + R_1},  \frac{r R_2}{K_{2i} + R_2} \right)
\end{equation}

Where $r$ is the maximal growth rate, $K_{ji}$ is the
half saturation constant for species $i$ limited by resource $j$.

To simplify the model we assume a tradeoff in resources requirement, with $K_{1i} = 1-K{2i}$, thus the
requirement of species $i$ for resource 1 is inversly related to the
requirement of resource 2 (the resource requirement is given by
$R_{ji}^*=\frac{m \, K_{ji}}{r-m}$, which is a common assumption
\citep{Tilman-1985,Schreiber-2003,Fox-2008}. **RICH I SAW YOU MADE THE
SAME ASSUMPTION ABOUT THE $C_{ji}$, I'm not sure this is needed, what
about if we set the $C=1$???**.


\begin{itemize}
\item As originally formulated, R* does not include traits
\item We adapt a formulation (due to whom?) where a ``trait'' is a
  measure of specialisation/generalisation.
\end{itemize}

\subsection{TREE}

In the second model (TREE) individual plants compete for light within a patch
and size-structured metapopulation \citet{Falster-2011, Falster-2015}.
Vegetation in this model comprises a large number of patches subject to a
prevailing disturbance. Disturbances remove establisghed vegetation within a
patch, which is then recolonosied via seed dispersal from other patches.
Indiivdual plants progress through their lifecycle, from seeds until mature
adults, with growth survival and fecundity rates determined by the combined
effect of an individual's traits, its current size and the degree of shading
from other plants within the same patch. The amount of shading at a given
height is computed from the cumulative leaf area above that height; short-term
competitive effects are therfore size-asymmetric. The sole effect of shading
is to alter an individual's photosynthetic income. Following
\citep{Falster-2015}, the fitness of individuals with a given type is
calculated by integrating seed output across the entire metapopulation.

We use the TREE model to estimate trait-based compeitive coefficients for
two traits whose evolutionary dyanmics have recently been investigated
\citet{Falster-2015}: leaf mass per area (LMA) and height at maturation
(HMAT). In the model, LMA moderates a physiological trade-off between the cost
of building an extra unit of leaf area and the rate of leaf turnover.
Variation in LMA thereby undeprins a trade-off between height growth rate and
shade-tolerance of seedlings. Meanwhile, HMAT moderates the allocation of
suplus energy between growth and seed production, as an indivdual increases in
size.

\clearpage

\section{Results}

\clearpage
\section{Discussion}

\textit{Main issues of competition models illustrated by our analysis}
\begin{itemize}
\item No clear generalisation yet about competition kernel shapes.
  \begin{itemize}
  \item our lma kernel looks a little like the
    \citet{Roughgarden-1979} offset Gaussian, but the ridges are
    likely important for later species (we need to actually show this
    however)
  \item our hmat kernel looks like the \citet{Kisdi-1999} asymmetric
    kernel, and the dynamics are fairly similar (needs
    demonstrating).
  \item The R* kernels just look weird, but that's because the
    \emph{trait} doesn't really map onto the resource very directly;
    the trait is the preference for two different resources, which
    can't map onto any single resource very well.
  \end{itemize}
\item Kernels other than Gaussian may lead to coexistance more easily
  than Gaussian.
  \begin{itemize}
  \item It's not all bad news; kernels like the lma kernel look
    ridiculous and you're not going to be allowed to propose them for
    a modelling paper.
  \item But they arise from size-structured dynamics in the model, so
    may not be that uncommon (worth noting that almost all the models
    are really of discrete exchangable adults and that size/age
    structure was flaged as an issue arges ago but I've lost the
    paper).
  \end{itemize}
\item Density dependent competition emerged in every model, as
  predicted by \citet{Abrams-1980}, who argued that not satisfying
  implicit assumptions would lead to widespread nonconstant density
  dependence.
\item More than density dependence, the \emph{shape} of the kernel
  changes with density, moving in position as the resident population
  density changes.  Kernels of this type simply haven't been studied
  and it's not actually very clear if kernels of this type might
  stretch the concept beyond its useful limit.
\item Outcome of competition is governed by the relative shapes of r,
  K and $\alpha$ but most modelling focusses only on the competition
  kernel, setting $K$ to be uniform or Gaussian and setting $r$ to be
  uniform.  The exception is asymmetric competition studies that tend
  to use asymmetric $r$ functions (pointing in the opposite direction
  of competitive ability). We don't know what \emph{any} of these look
  like in nature.
\end{itemize}

\textit{Traits are a productive way forward -- for what?}
\begin{itemize}
\item What did we want to say here, really?  Traits are already how
  people are thinking about kernels, but the work on resources hasn't
  caught up (see last point)
\item In a way, we're arguing to turn the way that we've looked at
  competition around: rather than asking ``given we know species
  coexist, what do we have to do to competition kernels'', we're
  saying: ``given we have data on species traits, can we just look at
  the shapes that competition functions actually take and use that to
  guide future modelling''.
\end{itemize}

\textit{(Maybe) link to field measurement of competition.}
\begin{itemize}
\item Most manipulative measures of competition in the field are going
  to give \emph{discrete} estimates of $\alpha$: the amount of
  competition exerted by or felt by a single species in the context of
  some community.
\item Because original, \citet{MacArthur-1967}, version of competition
  kernels was designed around resource utilisation functions, it
  should be possible to estimate competition coefficients and kernels
  from shapes of those functions if they were estimatable.
\item Does our way offer any better way forward?  The nonmanipulative
  work that Georges is doing with the working group data, or that
  James is looking at with the BCI data might help; we have large data
  sets of traits and growth; the signature of competition may show up
  there (perhaps we could put it on a manifold\ldots).
\end{itemize}

\textit{Trait-resource utility linkage is important, but largely
  ignored.  Mechanistic models provide some sort of way forward.}
\begin{itemize}
\item Of course we don't actually make this link in the model --
  perhaps we should
\item The 1:1 mapping happened very quietly, but the resource axis has
  gradually been entirely replaced by trait axes.  We argue that
  neither are likely to exhibit behaviour like the
  \citet{MacArthur-1967} model in natural systems.
\item It's hard to bridge between abstract models of competition that
  inform most theory with the sort of traits and species data that we
  have for plant communities.
\end{itemize}

\textit{Conclusions}
\begin{itemize}
\item Coming up with a positive note here might be hard.
\end{itemize}

\clearpage
\bibliographystyle{amnat}
\bibliography{references}

\end{document}

%%% Local Variables:
%%% mode: latex
%%% TeX-master: t
%%% TeX-PDF-mode: t
%%% End:
