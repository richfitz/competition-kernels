\documentclass[a4paper,11pt]{article}
\usepackage[osf]{mathpazo}
\usepackage{ms}
\usepackage{natbib}
\usepackage{graphicx}
\usepackage{caption}
\usepackage[labelfont=bf]{caption} % make label for figure bold

% Allow referencing into the supporting information, once that exists.
\IfFileExists{./competition-kernels-sm.tex}{%
  \usepackage{xr}%
  \externaldocument{competition-kernels-sm}}{}

\title{Competition kernels: shapes and consequences}
\author{Richard G. FitzJohn, Daniel S. Falster, Georges Kunstler and Mark Westoby}
\date{}
\affiliation{Department of Biological Sciences, Macquarie University,
  Sydney, Australia}
\runninghead{The shape of competition kernels}
\keywords{Competition, coexistence, kernels, resources}

\usepackage{color}

\newcommand{\ud}{\ensuremath{\mathrm{d}}}
\newcommand{\sign}{\mathop{\mathrm{sign}}\nolimits}
\newcommand{\Rstar}{\ensuremath{R^*}}
\newcommand{\plant}{{\tt plant}}
\newcommand{\hmat}{\ensuremath{h_{\text{mat}}}}
\newcommand{\TODO}{{\color{red}\sc todo}}

\newcommand{\figRstar}{3}
\newcommand{\figPlantHmat}{4}
\newcommand{\figPlantLma}{5}


\begin{document}

% Disable words breaking over lines for final submission:
% \raggedright
% \pagestyle{empty}

\mstitlepage
\parindent=1.5em
\addtolength{\parskip}{.3em}

\begin{abstract}
Abstract goes here\ldots
\end{abstract}

\section{Introduction}

% TODO: in one of the first three sentences: get across the "core
% question" of:
%
%   How do traits and resources (via competition) structure communities?
Ecological communities are structured by the nature of the
interactions within them;
% Wondering here if thisis a good place to cite Dolphs "Adaptive
% Radiations" book, as it gets close to some of this.
in particular competition for common resources structures niches by
excluding some species and allowing others to survive
\citep[e.g.][]{Gause-1934,Lack-1947,MacArthur-1967}.
% TODO: I have a paper *somewhere*, probably from Quarterly Review of
% Biology that asks (skeptically) the question of whether plants
% see/use fewer resources than animals do.
%
% NOTE: this is not great, but we need a solid link through to
% resources.
If species differentially consume different resources, or different
types of a given resource, this can regions of trait space that
different species can stably occupy and coexist.
%
The idea of species arranged along niche axes (e.g., Figure
\ref{fig:competition-kernels}) is iconic within community ecology and
is almost universally present in textbooks from 1970s through to the
present \citep[e.g.,][]{Krebs-1972, Ricklefs-1973, Ricklefs-1999,
  Krebs-2013}.
%
However, this worldview is particularly challenging for plant
communities, where most species use same few resources
\citep[e.g.,][]{Whittaker-1970, Grime-1979, Hubbell-2001,
  Westoby-2002}.

Currently, at least three perspectives are popular in investigating
the relationship between resources, species traits and coexistance:
%
i) Theoretical models based on explicit ``competition kernels'' and
simple Lotka-Volterra dynamics
\citep[e.g.][]{MacArthur-1967,Case-1982,Dieckmann-1999,Leimar-2013}.
%
ii) Models in which resources are modelled explicitly, and species
interactions emerge from competition for the resources
\citep[e.g.][]{Tilman-1977, Huston-1987, Kohyama-1993, Falster-2011}.
%
iii) Empirical, generally correlative, approaches that seek to
describe major axes of trait variation among species and relate that
to underlying differences in competition for resources
\citep[e.g.][]{Grime-1979, Weiher-1999, Westoby-2002}.
% NOTE: ignoring 4th approach for now, via ecological drift /
% neutrality, but it falls most closely into the 3rd approach.  Framed
% as above, in terms of "traits / resources / competition" neutral
% theory is not as relevant anyway.
Broadly these different literatures represent different ways of
addressing the same core question rather addressing fundamentally
different hypotheses.  Despite this similarity they have remained
largely isolated from one-another.

In the first strand, individuals from different species interact via a
``competition kernel'' --- a function describing how rates of
population growth of one species are reduced by the abundance of
another species with respect to resource preferences or traits of the
two species.
%
While initially competition kernels were derived via resource
utilisation curves, and expressed relative to a resource axis
\citep{MacArthur-1967, Roughgarden-1979}, there has been an almost
complete switch to expressing kernels based on difference in trait
values.
% 
Competition kernel--based models have been usd to model dynamics of
character displacement \citep{Taper-1985, Case-2000}, species
coexistence and community assembly \citep{MacArthur-1967,
  Calcagno-2006, Leimar-2013}, and sympatric speciation
\citep{Dieckmann-1999}.
%
These models are amenable to analysis and it is possible to determine
the outcome of competition from subtle changes to the competition
kernel \citep{Calcagno-2006, Scheffer-2006, Pigolotti-2007,
  Leimar-2013}.
%
In practice, researchers have focussed on a limited number of kernel
shapes, in particular Gaussian, asymmetric Gaussian or logistic shapes
(Table \ref{tab:kernel-shapes})

Resource based models differ from the kernel-based approaches in that,
rather than being inferred from a trait-based kernel, competitive
interactions are moderated via a resource whose abundance is modelled
explicitly.  As such, the competitive effects ``emerge'' from the
model, and the shape of competitive interactions with respect to
species traits is not specified.
%
In this paper, we will consider two classes of resource-based model.
%
The first is Tilman's ``\Rstar'' model \citep{Tilman-1977,
  Tilman-1982}, which focusses on the minimal level of resource
abundance that individuals from a species can tolerate.
% Direct species intereractions are absent, and resources do not
% instantly replenish.  Trait link is low.
%
% I don't think this matters here?:
%
% Similar to the kernel-based models, populations are unstructured, so
% competitive interactions occur within a ``mean field''.
The second type are mechanistic niche models, which allow for size-and
trait-driven interactions within size-structured communities or
meta-communities
\citep{Huston-1987,Kohyama-1993,Moorcroft-2001,Falster-2011}.
% Dropped:
% Although competition for below ground resources may also be
% considered, competitive occur largely via shading of smaller plants by
% taller plants within a patch.
% ----
% also models in which individuals compete for space. But these
% actually fall in with kernel based models - competition-colonisation
% models
% In both types of model, traits modify how species use the resource, but to the
% extent they exist,  any species-level competition-kernel arises only as emergent property
% of the entire system.
% Extra sentence - perhaps not needed
% Beyond the handling of competitive interactions, the dynamics of character evolution
% and species coexistence are modelled in largely the same way as in kernel-based models
% \citep[e.g.][]{Falster-2015}.

The empirical approach to investigating how species traits affect
coexistence attempts to classify species into different strategy
schemes
\citep{Grime-1974,Grime-1979,Southwood-1977,Weiher-1999,Westoby-2002}.
%
% For plants, prominent traits include morphological and physiological
% characteristics of individual tissues such as leaf mass per area
% (\textsc{lma}), and features affecting allocation, such as height at
% maturation (\textsc{hmat}) \citep{Westoby-2002}.
%
Existing strategy schemes lack explicit theory illustrating how
trait-based community might work \citep{Adler-2013}.
% TODO: this feels weak still:
However, characteristics such as trait over-dispersion may be
signatures of niche diversification \citep[e.g.][]{Kraft-2008} while
correlations amongst traits and growth rates may indicate
diversification along successional niche \citep[e.g.][]{Wright-2010}.
%
The validity of these inferences depends on the insights drawn from
the two more theoretical approaches.

Ideally, these different approaches to studying community assembly
would mutually reinforce each other, yet to date the literatures
remain relatively isolated from one another.
%
In this paper we try to bridge this divide, by asking whether the
signature of competitive interactions that emerges from resource-based
models reflects the trait-based kernels used in kernel-based
approaches.
% 
To achieve this we determine the shape of the competition functions
that are produced in two different models where competition and the
mechanics of the resource being depleted are represented explicitly.
%
This approach is not new; \citet{Abrams-2008} found that \Rstar\ model
satisfies few of the assumptions that underpin Lotka-Volterra
dynamics.
%
We extend this approach by focussing on trait-based kernels and by
including a second class of resource-based model involving
successional niche differentiation.
%
Specifically, we assess whether several assumptions embedded within
most kernel-based analyses are compatible with the behaviour of the
resource-based models and ask;
%
i) what is the shape of the competition function emerging from resource-based
models?
%
ii) does the shape of the competition function remain constant with
respect to the traits of species in the community?
%
iii) does the shape of the competition function remain constant with
respect to the density of the species in the community, as commonly
assumed in kernel-based models?

\section{Models}

In general, we would like to know what the shape of competition
functions look like in natural systems.  However, holding all else
constant and isolating the effect of one species on the population
growth rate of another is intractable.
%
An alternative approach is to analyse models where the competitive
effects emerge from lower-level interactions
\citep[e.g.,][]{Draghi-2012}.  Mechanistic models make more direct
connections to the natural world than the phenomenological models,
while still remaining tractable.  This does not require or assume that
the mechansitic models are correct or complete descriptions of the
natural world.

% NOTE: Thoughout, decide how to use the terms ``competition kernel''
% and ``competition function''; in the usual use they are roughly
% equivalent but here they are probably not: see
% http://math.stackexchange.com/questions/22133/difference-between-kernel-and-function
% That post doesn't really clarify much, in part because there are multiple definitions
% of kernel http://en.wikipedia.org/wiki/Kernel
% This one seems most useful: http://en.wikipedia.org/wiki/Equaliser_(mathematics)

% In the first paragraph, I guess we describe the entire approach as
% well as possible.

% NOTE: This is hard to introduce properly because the Abrams kernels
% need dealing with explicitly.  But of course we don't really use
% them (yet -- see issue #35), and note that not all are even possible
% (some require that coexistance is possible).
We investigate the dynamics of two quite different models --- \plant\
and \Rstar\ --- where the competitive impact of one species on another
arises via the effect of individuals on a shared resource.
%
In each model, species traits influence the dynamics of resource use,
which in turn affects competitive interactions through the common
resource.  Several ways of computing competition coefficients are
possible \citep[e.g.,][]{Abrams-1987,Abrams-2008}.  When the
underlying dynamics follow Lotka-Volterra dynamics (density
independent competition, etc), then the different coefficients will be
the same.  We use another formulation based on simply solving for the
competition coefficient in the Lotka-Volterra equation.

For each model, we compute the core quantities used in the
Lotka-Volterra equations (growth rate, carrying capacity, per-capita
rate of population increase and strength of competition).  The
standard Lotka-Volterra equations express the rate of increase in the
abundance of species $i$ ($N_i$) as a product of two terms.
%
The first is the maximum possible per-capita growth rate, $r_i$. The
second expresses the degree to which that maximum is realised, and
thus includes any reductions due to competition with other species. A
commonly used form for Lotka-Volterra dynamics is
%
\begin{equation}
\label{eq:LV}
\frac{\ud N_i}{\ud t} =
N_i r_i \left(1 - \sum_j \frac{\alpha_{ij} N_j}{K_i}\right),
\end{equation}
%
where $K_i$ is the equilibrium value for $N_i$ when species-$i$ is
grown in a monoculture (i.e., the carrying capacity), and
$\alpha_{ij}$ is the per-capita competitive effect of species-$j$ on
species-$i$.  By convention, competition functions are expressed as
the effect of species-$j$ on species-$i$, relative to the effect of
species-$i$ on itself, so $\alpha_{ii}\equiv 1$, so an $\alpha_{ij} >
1$ implies stronger iterspecific competition than intraspecific
competition.  The sum is over all species in the population, with
competitive effects assumed to be additive across species and density
independent within a species.
%
Other forms of equation (\ref{eq:LV}) exist that differ in the nature
of density dependence \citep[e.g.][]{Leimar-2013}, but for simplicity
consider a single version.

We are interested in situations where the different parameters in
equation (\ref{eq:LV}) are functions of traits; that is replacing the
subscripts in the above equations as continuous functions of species
traits $x$.  In particular the coefficients $\alpha_{ij}$ becomes a
function $\alpha(x_i, x_j)$.
%
We restrict our analysis to the simplest situation where a single
invading type with trait $x_I$ and a community comprising a single
resident type with trait $x_R$ at equilibrium.  The fitness $f_I(x_I,
x_R)$ of the invading type is its per capita rate of increase when
rare ($N_I = 0$), or:
%
\begin{equation}
  \label{eq:LVi}
   f_I(x_I, x_R) = \frac{\ud N_i}{\ud t} \frac{1}{N_i} =
    r(x_I)\left(1 - \alpha(x_R, x_I) \frac{K(x_R)}{K(x_I)}\right).
\end{equation}
% TODO: Move this into the discussion, I think.  This is the
% "classical" criticism of LV as a useful device for doing anything in
% natural systems.
%
%   The reason for this narrowing of focus is that it is not possible to
%   infer the shapes of competition functions in situations with more
%   than one resident, because the number of unknowns increases faster
%   than the number of species (refs?).

To estimate the shapes of competition functions embedded within the
\Rstar\ and \plant\ models, we use those models to calculate all of
the trait-dependent terms in eq. \ref{eq:LVi} except $\alpha(x_R,
x_I)$:
\begin{enumerate}
\item $f_I(x_I, x_R)$: fitness of rare invader with traits $x_I$  in
  competition with a established resident with traits $x_R$
\item $r(x_I)$: maximum possible fitness of invader, i.e. fitness in
  the empty environment.
\item $K(x_I)$: equilibrium population density of invader when growing
  in monoculture.
\item $K(x_R)$: equilibrium population density of resident when
  growing in monoculture.
\end{enumerate}
We then solved for $\alpha(x_R, x_I)$ by rearranging  \ref{eq:LVi} to give:
\begin{equation}
  \label{eq:alpha}
  \alpha(x_R, x_I) = \frac{K(x_I)}{K(x_R)}
  \left(1 - \frac{f_I(x_I, x_R)}{r(x_I)}\right).
\end{equation}

In addition, we will also use one of the coefficients described in
\citet{Abrams-2008}: $\chi_{ij}^{*}$:
%
\begin{equation}
  \label{eq:abrams-coef}
  \chi_{IR}^{*} =
  \frac{\partial f_I(x_I, x_R, N_I, N_R) /  \partial N_R}%
  {\partial f_I(x_I, x_R, N_I, N_R) / \partial N_I}
  \bigg|_{N_I=0; N_R=K_{R}}
\end{equation}
%
where the numerator is the derivative of invasion fitness with respect
to resident density, the denominator is the derivative of invasion
fitness with respect to invader density, and this is evaluated at the
situation where the resident is at equilibrium and the invader is at
trivial density.  This index therefore measures the extent to which
the resident depresses the growth of an invader relative to the amount
that the invader depresses its own growth.

\subsection{\Rstar\ model}

% \item As originally formulated, \Rstar does not include traits
% \item We adapt a formulation (due to whom?) where a ``trait'' is a
%   measure of specialisation/generalisation.

In the first model (\Rstar) individuals compete for two essential and
depletable resources. This model is one of the best studied and cited
models of resource competition \citep{Tilman-1977, Tilman-1982,
  Huisman-2001}, originating from a study on exploitative competition
for any pair of generic resources \citep{Leon-1975}. The model is
unstructured, with the per capita growth rates for each species simply
depending on the availability of two resources.  We assume both
resources are necessary (i.e. non-substitutable), so the growth rate
of each species is determined by the resource which is currently most
limiting.  We implemented a version where the specific growth rates
follow a Monod equation as in \citet{Huisman-2001}, which is closely
similar to the widely-used version described by \citet{Tilman-1977,
  Tilman-1982}.  The resource availabilities, in turn, depend on the
rates of resource supply and consumption. Individuals compete with one
another by drawing down the joint resource pools. Such a system is
easily described via a minimal set of equations (Appendix
\ref{sec:Rstar}).

Following previous authors \citep{Tilman-1985, Schreiber-2003,
  Fox-2008} we focus on a trait determining the relative amount of the
two resources required for growth. Species can vary from requiring
mostly resource 1, to equals amounts of both resources, to mostly
resource 2. This dependence is achieved by making uptake saturation
constants (see appendix \ref{sec:Rstar} for details) for resource 1
inversely related to the requirement of resource 2:
\begin{equation}
\label{eq:R3}
U_{1i} = 1 - U_{2i}.
\end{equation}
(Note that the $U$'s here are $K$'s in \citet{Huisman-2001}, but we
use $U$ to avoid confusion with carrying capacity.)
%
The rate of consumption of resouce $j$ by species $i$ is determined by
$c_{ki}$ the content of resource $k$ in species $i$. We assumed that
the content was similar across species.

% GEORGES COMMENT IF WE KEEP USING THIS FORMULATION WE NEED TO EXPLAIN
% C HERE OTHERWISE IT SHOW UP IN THE RESULTS WITHOUT ANY
% DEFINITION. IF WE ASSUME C=U THIS MAY HELP TO SIMPLIFY THIS SECTION?
%
% Yes - I (RGF) agree - just need to port the model over to use the
% formulation that you prefer.

\subsection{\plant\ model}

The second model, \plant, is a mechanistic model of plant competition
for light in a size-structured metapopulation \citep{Falster-2011,
  Falster-2015}. \plant\ is used here as an example of a broader group
of models dealing with successional turnover
\citep{Huston-1987,Kohyama-1993,Moorcroft-2001,Falster-2011}.

Vegetation in the \plant\ model comprises a large number of patches subject
to a prevailing disturbance regime.  Disturbances remove established
vegetation within a patch, which is then recolonised via seed
dispersal from other patches.  Individual plants progress through
their life-cycle, from seeds until mature adults, with growth survival
and fecundity rates determined by the combined effect of an
individual's traits, its current size and the degree of shading from
other plants within the same patch. The amount of shading at a given
height is computed from the cumulative leaf area above that height;
short-term competitive effects are therefore size-asymmetric. The sole
effect of shading is to alter an individual's photosynthetic
income. Following \citep{Falster-2015}, the fitness of individuals
with a given type is calculated by integrating seed output across the
entire metapopulation.

We use the \plant\ model to estimate trait-based competitive functions
for two traits whose evolutionary dynamics have recently been
investigated \citep{Falster-2015}: leaf mass per area (\textsc{lma})
and height at maturation (\hmat). In the model, \textsc{lma} moderates
a physiological trade-off between the cost of building an extra unit
of leaf area and the rate of leaf turnover.  Variation in \textsc{lma}
thereby underpins a trade-off between height growth rate and
shade-tolerance of seedlings, and is a key comonent of the so-called
``leaf economics spectrum'' \citep{Wright-2004}. Meanwhile, \hmat\
moderates the allocation of surplus energy between growth and seed
production, as an individual increases in size.

% TODO: Something about how hmat leads to a fundamentally asymmetric
% competition situation? -- i.e., talk about the *expectations* for
% the different models.

\section{Results}

% For each model there are a few things to discuss:
%
% - Non-equilibrium before equilibrium
% - Density dependence after discussing each model
% - Different measures of competition (where?)

\subsection{\Rstar}

For the \Rstar model we consider two cases: a ``symmetric
consumption'' case with $C = \{0.2, 0.2\}$ and an ``asymmetric case''
with $C = \{0.3, 0.7\}$.  Species vary only in their rate of resource
conversion efficiency and the degree in which they specialise in the
two resources.  For each case, we consider two resident communities;
one in a position where invasion by other types is possible and one at
the evolutionary optimum point.
%
Because $C$ does not vary by species, coexistence is not possible in
this model \citep{Tilman-1982}; invasion is a sufficient condition for
competitive replacement.

In the symmetric model, we first consider a resident at $x = 0.6$
(midway between pure generalist and pure specialist on resource B).
%
The resident species competes strongly with species that are more
specialised than itself (self competition is $\alpha(x, x) \equiv
1$), while invaders that were less specialised (i.e., closer to $x =
0.5$ than the resident) experienced less competition from the resident
than it exerted on itself (Figure \ref{fig:Rstar}b).
%
The region of competition $< 1$ lines up perfectly with the region
where invasion is possible (black bar on Figure \ref{fig:Rstar}b,
fitness greater than zero on Figure
\ref{fig:sm_rstar_components1}c1).
%
When the resident is moved to $x = 0.5$ (perfectly generalist) then
invasion is not possible for any type.  In this case, the strength of
competition increases monotonically away from the resident.
%
Computing competition functions using the derivative approach
described in \citet{Abrams-2008}
($(\partial d_i / \partial N_j) / (\partial d_i / \partial N_i) |_{N_i
  = 0, N_j = K_j}$)
was qualitatively similar, though picewise linear in species traits
and with higher departure from 1.

In the asymmetric case, consumption rates are higher for
resource $B$ than resource $A$; this shifts the model in favour of
species that can most efficiently use resource $B$.
%
With a single resident at $x = 0.7$, invasion is possible for species
that use $B$ more efficiently ($x$ in $(0.7, \approx 0.99)$).
%
The shape of the competition function was considerably more
complicated than the symmetric case, though the region of invasion
still corresponded to the region where competition was less than 1.

In both the symmetric and asymmetric case, competition was density
dependent.  Increasing the resident density changes the per-capita
competitive impact (this is not the case for models like the
\citet{Dieckmann-1999} model where the per-capita competitive effect
stays constant).
%
In the symmetric model, the \emph{shape} did not qualitatively change,
and competition was always saturating with increasing density.
Increasing species density decreased the per-capita competitive effect
(Figure \ref{fig:rstar_density_dependence}a--b).
%
In the asymmetric model, the shape of competition qualitatively
changed, though on average per-capita competition appeared to decrease
with increasing density (Figure
\ref{fig:rstar_density_dependence}c--d).

\subsection{\plant: \hmat}

In the \plant model, Strategies that have different height of
maturation (\hmat) begin to focus resources on reproduction at
different ages.  Plants that reproduce earlier get to ``lock in''
reproduction, but because they allocate less resources to growth they
risk being over-topped and out-shaded by strategies that reproduce
later.  This leads to a strongly asymmetric competitive effect of
taller strategies suppressing the population growth of shorter
strategies.

We recover competition functions that are asymmetric in this case:
strategies exert more competition on species that are shorter than
them than are taller than them (Figure \ref{fig:plant_hmat}).  As the
resident strategy increases in size, the strength of competition felt
by smaller species increases, and the slope of this competitive effect
increases (compare panels \textit{a} and \textit{b} in Figure
\ref{fig:plant_hmat}.
%
Despite the consistent direction of competition, invasion depends on
the interaction between competitive effect, carrying capacity and
maximum growth rate; in the presence of a short resident, taller
strategies can invade (Figure \ref{fig:sm_plant_hmat_components}\,c1),
while in the presence of a tall resident, shorter strategies can
invade (Figure \ref{fig:sm_plant_hmat_components}\,c2).

This property is very similar to other models of asymmetric
competition \citep[e.g.,][]{Kisdi-1999}, where selection pushes traits
to an ``branching point'' (close to the resident in (Figure
\ref{fig:sm_plant_hmat_components}c2), and then repeated invasion to
the left of this point is possible.
% TODO: SM figure showing the equilibrium community for each of the
% scenarios presented (#31).
Even though the shape of the function is not exactly sigmoidal, the
evolutionary dynamics are similar (\TODO).
% TODO: Run out the evolutionary dynamics of plant/hmat and Kisdi to
% show that things do actually look similar with continual reinvasion
% of mutants to the left.
% TODO: density dependence
% TODO: Abrams kernel calculation

\subsection{\plant: lma}
Strategies that have different leaf mass per unit area (\textsc{lma})
values are different points in the ``leaf economic spectrum''
\citep{Reich-1997,Wright-2004}.  We modelled \textsc{lma} as a trade-off
between the cost of deploying a unit of cost of building an extra unit
of leaf area and the rate of leaf turnover (so high \textsc{lma} leaves are
costly to deploy but turn over more slowly so that they can contribute
to carbon gain for longer, or conversely require rebuilding less
often).

The shape of the competition function is complex and does not conform
to any of the simple shapes in wide use.
% Need to be clear that we're talking about the non attractor
% situation first.
In the region of trait space close to the resident species,
competition is strongly asymmetric: the resident exerts a large amount
of competition on species with \textsc{lma} values higher than itself (``slow''
strategy species) and less on species that have \textsc{lma} values lower
(``fast'' strategy species; Figure \ref{fig:plant_lma}a).  Invasion is
possible in the region with \textsc{lma} values smaller than the invader until
growth and carrying capacity reach a biological limit to invasion
(Figure \ref{fig:sm_plant_lma_components}c1).
%
Above the resident \textsc{lma} value, the competition function is sawtooth
shaped, before eventually declining steeply.
%
In contrast to the complex shape of fitness and competition, maximum
growth rate has a simple form and carrying capacity is approximately
Gaussian (Figure \ref{fig:sm_plant_lma_components}\,a and b).

Moving the resident towards the evolutionary attractor, invasion is
possible through a broad range of trait values (both avove and below
the resident value).
%
Make the case that the sawtoothness \textbf{probably} corresponds to
differences in how time is partitioned (\TODO).
%
Explain how is it the case that invasion is possible through the
region where competition $\approx 1$ -- that doesn't make much sense!
It's similar to the situation with \hmat\ though (\TODO).

\section{Discussion}

We inferred the shape of trait-based competition functions embedded within
two resource-based models of competition, focussing on the special (but
important) case of a single resident preventing or allowing invasion of a new
phenotype by competing with it. In the models we looked at, the competition
functions: i) were density dependent, implying non-linear per-capita
competitive effects \citep{Abrams-1980}; ii) varied qualitatively and
quantitatively with resident phenotype, implying that they are not actually
kernels but functions, and iii) had shapes that were distinctly non-Gaussian,
suggesting limited connection with the wide array of theoretical models
assuming Gaussian shapes. Trivially, our results demonstrate that the \Rstar
and \plant  models are not well approximated by a Lotka-Volterra system, at
least for two of the three traits we investigated. That the kernel-based models
do not perfectly approximate the resource-based models in itself  is not
surprising: it should certainly be hoped that more complex resource-based models
are in some way richer that the standard LV equations. More substantially, the
departure in the derived competition functions from the Gaussian kernels
assumed by most theoretical investigations is so drastic, it is difficult to
reconcile the two approaches. Both approaches purport to give insights into the 
mechanisms of community assembly via competition for resources. Yet, there seems
no straightforward way of interpreting the results from the resource-based
models, for two of the three traits investigated, within the known behaviour
of LV systems. This perhaps suggests that we have been focussing our
theoretical investigations on an unnecessarily restricted region of possible
model space.

\subsection{The history of kernel-based approaches}

\emph{``The nearly complete dependence of theoretical community ecology on a
limited number of simple models has resulted in the development of subsequent
generations of deceptively simple constructs that may often confuse our
understanding of the underlying biology rather than improve it. The 
Lotka-Volterra competition equations and the community matrix are one such 
pair of constructs.''} \citep{Neill-1974}

As the above quote indicates, the dominance of a single model type within
theoretical ecology was already of concern in 1974. In subsequent 40 years,
the only substantial change has been the widespread adoption of trait-based
competition kernels. Given  the dominance of this approach within this branch,
it may be helpful to review the history by which the which this dominance came
about.

Central to community ecology has been the metaphor of species packing. The
community is seen as fitted into a volume of niche space. Each species that is
a potential community member has a fundamental niche somewhere in the volume.
The fundamental niche is compressed by competition into a realised niche, as
species are packed in alongside each other (Figure \ref{fig:competition-
kernels} for one-dimensional version found in [almost all] textbooks). Roots
of the species packing metaphor trace to the competitive exclusion principle
\citep{Lotka-1925, Volterra-1926, Gause-1934} whereby if species have
identical niches or compete for a single resource, all but one are
competitively excluded. \citet{Hutchinson-1957} added the n-dimensional hyper-
volume. The competitive exclusion principle was thereby converted into the
question how  dissimilar two species need to be in order to coexist
\citep{MacArthur-1967, Slatkin-1980}. By the 1970s, a new generation of
textbooks built their account of community ecology around three alternative
ways of getting more species into a community: larger total niche space, more
overlap between adjacent niches, or narrower niches for each species
\citep{Whittaker-1970,Krebs-1972,Ricklefs-1973,Pianka-1974}. The species
packing  metaphor has persisted up to the present day as the mainstream
approach to  community ecology. (Indeed,  \citet{Krebs-2013},
\citet{Ricklefs-1999} and \citet {Pianka-1999} continue also.)

In 1960s theory and 1970s textbooks, niches were typically thought of as
utilization functions along a continuous resource axis. The resource in
question would be depletable or pre-emptable, in order for overlap of
utilization functions to translate into competition. Pairwise competition
coefficients (such as alphas in Lotka-Volterra competition) could be
calculated from overlap of utilizations functions. If a trait was used as a
niche index, it was chosen on grounds of mapping tidily on to a depletable
resource, such as beak size for seed-eating finches. Beginning in the 1980s,
a desire to model character displacement saw the concept of a utilisation function
first approximated via a trait -- the mean size of utilisation function \citep
{Roughgarden-1979} --  and second replaced entirely by a kernel defined by traits
\citep{Slatkin-1980,Taper-1985}. This shift enabled and flowed into
models for character (= trait) displacement \citep[e.g.,][]{Taper-1985,
Case-2000, Goldberg-2006} and trait evolution
\citep{Brown-1987,Geritz-1998,Geritz-1999,Dieckmann-1999, Abrams-2001}. 

Trait-based competition kernels are best thought of within the adaptive-
dynamics tradition of fitness landscapes. Species exist, coexist and evolve on
a landscape where the axes are evolvable traits. A competition kernel
represents the impact of a species on this landscape -- the drawdown of
fitness across the landscape as it relates to the trait-values both of the
species exerting a competitive effect and of the species experiencing the
effect. As such, the shape of the kernel is central
to the question how species interact within the models using them.
 By far the most common type of
competition kernel in textbooks and in theoretical literature is that they are
unimodal (Gaussian), decreasing with trait-distance away from the species that
is exerting the competitive effect. Other shapes have been considered,
but for the most part these are variants on two forms: Gaussian and logistic
(Table \ref{tab:shapes}). Surprisingly little is known about
alternative possible shapes for competition kernels and their consequences for
community assembly.

In parallel, two other related strands of thinking that persist to current
day were established. In a break from the idea of subdividing continuous
niche axes, models arose
that visualised communities as a chessboard of living sites, each able to be
occupied by one sessile individual \citep{Yodzis-1978}. Coexistence could
arise by colonization-competition trade-offs \citep{Levins-1971,
Geritz-1999, Calcagno-2006}; which required that competitive interactions
be strongly size-asymmetric. Alternatively, if there is an advantage to 
already established individuals (another form of competitive asymmetry),
then chance effects in the occupation of vacant living sites can slow
competitive exclusion by pure demographic stochasticity (``lottery
effects'')
\citep {Sale-1977, Hubbell-2001} or through environmental fluctuations
(``storage effects'') \citep{Chesson-1981}. While expressed differently,
it is important to note the models are compatible with kernel-based 
approaches -- the first using sharp logistic kernel and the second a 
flat, trait-independent kernel. 

The other strand of thinking, already established during the 1970s in parallel
with the species packing metaphor, was theory showing that with Gaussian
kernels along a single niche axis, there is in fact no limit to similarity
\citep{May-1972, Abrams-1975, Abrams-1983}. Competitive exclusion can become
slow because two competitors are similar in their resource use as well as
because they are different. ``Continuous coexistence'' is a theoretically
possible outcome, with resources occupied by many species spaced from each
other in resource use by very small amounts. Although this strand of theory
might have been expected to undermine the mainstream approach to community
ecology via the species packing metaphor, in actuality it did not have the
effect of dislodging species packing from standard textbooks, probably because
it did not really offer an alternative program for connecting theory to field
measurement.  After 2000, this line of theory began to explore more
systematically the consequences of possible kernel-shapes for community
assembly \citep{Scheffer-2006, Barabas-2013, Leimar-2013}.

In summary, the current situation is that it is clear competition kernels on
fitness landscapes could be key to understanding community assembly. Yet, it
is also clear that symmetric uni-modal or Gaussian kernels are realistic for
only a minority of real-world communities, and the menagerie of possible
competition kernels has only begun to be catalogued (Table \ref{tab:shapes})
\citep{Barabas-2012,Leimar-2013} 
Much less have the
implications of different kernels for community assembly been worked through
\citep{Barabas-2012,Barabas-2013,DAndrea-2013,Leimar-2013}.
The literature up to the present has mainly been focused on Gaussian and
logistic kernels, perhaps because they were mistakenly thought to be
prevalent. Surprisingly little attention has been paid to the question what
shapes are actually likely for competition kernels, given known mechanisms for
resource depletion in relation to traits. We argue that clarifying plausible
shapes for competition kernels and their consequences will be an essential
foundation before a robust community ecology can be constructed.

\subsection{Competition functions in resource-based models look nothing like
fixed Gaussian kernels}

In only one of the three cases studied here did our derived competition
functions closely resemble any of the three dominant curves used in most
theoretical investigations. The common forms are the Gaussian, asymmetric
Gaussian \citet{Roughgarden-1979} and Logistic \citep[e.g.][]{Kisdi-1999,
Law-1997}. The derived function from the \hmat\ model looks a lot
like the logistic kernels used in models investigating asymmetric competition
\citep{Kisdi-1999, Law-1997, Geritz-1999,
Calcagno-2006}. The competition function for our \textsc{lma} model looks
vaguely resembles the asymmetric Gaussian, but the ridges along the top are
distinctly non-Gaussian and arise from real biological features of the system. 
(\citet{Falster-2011} showed how the amount of shading undulates through 
successional time. here this translates into undulations in fitness with
respect to traits.) It is hard to generalise about these findings; what we
can say is that none of the functions even closely resemble straight Gaussian
kernels. 

We also found that, for all three of the models, the derived competition
functions were density dependent. It was not only the amplitude of the
function that changed, but also its \emph{shape} (Figs XX). This result was
predicted by \citet{Abrams-1980}, who argued that not satisfying the implicit
of LV dynamics would lead to widespread non-constant density dependence. The
idea that competitive effects are non-linear in density (i.e. non additive) is
widely established within other more empirical realms of plant ecology, e.g.
in the self thinning law \citep{Westoby-1984}, and the law of constant final
yield \citep{Weiner-2010}. In both cases, the dynamics mirror are similar to
those we observe: initially effects in either biomass or competition increase with population
density (as would be expected with a constant per capita effect), but then
once a saturating population density has been reached, addition of more seed
has little impact on the community. As such, the per-capita effects changed
sharply with density. Similar findings
exist for animals: already by 1980 there were  multiple reports showing that
competition coefficients were not independent of population density
\citep{Abrams-1980}.

Together, these findings imply the deeply ingrained assumption that
competition kernels can be defined by differences in trait values
alone -- without reference to population size, or absolute value of traits
-- is simply not defensible for the resource-based competition in plant
communities. And by implication, the notion that a fixed competition
coefficient exists as a property between two species surely also 
disintegrates. This raises the question of whether we should be seeking to
revise our ideas about competition kernels, or more generally about the
usage LV equations as a foundation for modelling biological phenomena. 
Indeed, multiple authors have repeatedly raised such questions 
\citep[e.g.][]{Andrewartha-1953, Neill-1974, 
Abrams-1975, Wangersky-1978,Abrams-1980, Tilman-1987}. 

It is certainly possible to expand the idea of a competition kernel to be a
function not only of trait differences, but also of population density and
species traits. The consequence of including functions of this type in
traditional analyses about species-coexistence and character displacement are
simply unknown. However, such an approach would only work if the competition
functions derived from pairs of species could still be added to get the effect
of an entire resident community on a potential invader. Our results suggest it
such additivity is unlikely (Figs. ???). So in that case the competition
kernel would be a function of the entire resident community, perhaps operating
via the resource densities. At that point we are perhaps stretching he concept
beyond its useful limit. We may as well just use a model with explicit
handling of resources.

\subsection{Enhancing the scope of resource-based models}

\emph{``The analysis of the Lotka-Volterra model developed here can only
suggest that, if there are limits to similarity in natural systems, these
limits will depend sensitively on the form of the competition coefficient,
which must be derived from another model.''}\citep{Abrams-1975}

Terrestrial plant ecology had always found it difficult to accept the idea
that different species can only coexist by using different resources, since
plant species clearly all require the same sources
of light, soil water and mineral nutrients as each other. One
possibility arises from species subdividing a successional-time niche axis in
landscapes envisaged as disturbance mosaics \citep{Connell-1978}. If plant
species make most contribution to their population increase in different years
following disturbance, then they certainly are depleting different light and
water from each other, and quite possibly different soil nutrients
\citep{Moorcroft-2001}. Another possibility might arise via
differentiation by soil depth and by chemical forms of nutrients \citep{Tilman-1977}.

One of the primary reasons we used models to study the nature of competition
functions was due the difficulty of estimating these in the field.
Interestingly, \citet{Ricklefs-1973} suggests one might use the exact  method
we have applied to estimating competition functions, i.e. by measuring the
different elements of eq. \ref{eq:LVi} then using these to estimate the
competitive interaction (Fig. \ref{fig:components}). And indeed some estimates
have been made \citep{Ricklefs-1973, Kraft-2015}. However, nearly all
manipulative measures of competition in the field are going to give
\emph{discrete} estimates of $\alpha$: the amount of competition exerted by or
felt by a single species in the context of some community. However, as the
number of species increases, the number of pairwise competition coefficients
needed increases quadratically with the number of species. This is one reason
for seeking to define competition kernels on the basis of trait differences,
so as to reduce the number of parameters needed \citep{Kraft-2015}. However,
estimating the shapes of trait-based competition kernels requires that lots of
pairwise comparisons to made, which introduces new issues. Species usually
differ in multiple traits, plus the need to estimate fitness across the life-
cycle makes the task nigh impossible. Models are attractive because they allow
us to run experiments in ways that are simply not possible in real systems.

The nature of the competition functions generated by the resource-based models
studied here raise an exciting possibility: we could substantially  expand the
range of competition functions currently being considered (Table
\ref{tab:shapes}) and thereby expand possibilities of coexistence. Functions
like the \textsc {lma} kernel may look somewhat ridiculous for those used to
clean stylised analyses, yet they  arise naturally from size-structured
dynamics and so may in fact not be that uncommon.

\subsection{Future directions \& conclusions, where next with models of
competition}

Despite being difficult to quantify in the field and represent in models, 
competition is undoubtedly an important force in structuring communities.
Even neutral models, where the effect traits is removed, makes a strong
assumption regarding competition - that total abundance in the community
is fixed. Our results reinforce previous arguments 
\cite{Andrewartha-1953, Neill-1974, Abrams-1975,
Wangersky-1978,Abrams-1980, Tilman-1987} that if competition is density
dependent, non-additive across species, or if the shape of competition
varies with species traits and not just distance, then the Lotka-Volterra 
equation will be not adequately capture the dynamics of real systems. It's 
difficult see what a new set of equations might look like, however, as a
first step we might more vigorously focus on model where resources are
handled explicitly. 

Simultaneously, we should be looking to derive macroscopic approximations 
of the more detailed resource-based successional models. While we have
perhaps sounded rather negative of simple models, this is not meant to be 
a criticism of simple models \textit{per se}; simple models are highly
desirable. However, we argue that the simple models should be derived 
to approximate the behaviour of more complex models and empirical findings
\citep[e.g.][]{}.
% History of deriving higher-level approximations to detailed models. 
The problem with the LV system of equations is that the model was formulated
in isolation from the problem's to which it is now being applied.  

Along those lines, it seems productive to further investigate the nature of
competitive interactions as it relates to traits, but with increased emphasis
on the nature of resource that is being contested. In the case of traits such
as lma and wood density, differentiation corresponds to a subdividing of
successional time. There is also valuable insight to gained in applying the
idea of trait-based kernels to process  occurring over shorter time-frames,
i.e. in growth or survival process rather than lifetime fitness
\citep[e.g.][]{Kunstler-2012, Lasky-2015}.  The findings of such analyses test
more directly the assumptions entering into resource-based models and thereby
allow for tighter integration of theory with the sort of traits and species
data that we have for plant communities.

\clearpage

\section{Tables}


\begin{table}[h]
 \caption{Predominant kernel shapes used to date. 
`Kinked'' kernels that are ``nondifferentiable at the point of
self-competition'' \citep{Barabas-2012}.}
\centering
{\footnotesize
  \begin{tabular}{p{3cm}p{4cm}p{3cm}p{4cm}}
  \hline
  Name & Shape & Characteristic equation & References \\
  \hline
  \multicolumn{4}{l}{\textbf{Continuous}} \\
  Constant & & & Hubbell-2001, Egas-2004 \\ \\
  Gaussian & & & Slatkin-1980, Taper-1985,
  Dieckmann-1999\\ \\
  Platykurtic & & &  Leimar-2013 \\ \\
  Offset Gaussian & & & Slatkin-1980, Rummel-1985\\ \\
  ??? & & $1 + \exp(-(x_i + x_j + b)^2) - exp(-b^2)$ & Brown-1987-66 \\ \\
  Exponential & & $\exp(dx^4)$ &Pigolotti-2007, Leimar-2013\\ \\
  Logistic & & & Law-1997, Kisdi-1999, Geritz-1999, Calcagno-2006\\ \\
  Bessel & & & Bolker-2009\\ \\
% more from competition-colonisation literature?
 \multicolumn{4}{l}{\textbf{Kinked}} \\
  Laplacian & & $\exp(-2|x|)$ & Roughgarden-1972, Leimar-2013 \\ \\
  Gaussian with hat & & & Scheffer-2006, Barabas-2013, Leimar-2013 \\ \\
  Asymmetric step & & & Kinzig-1999, Calcagno-2006, DAndrea-2013 \\
  \hline
  \end{tabular}
}
\label{tab:shapes}
\end{table}

\clearpage

\section{Figures}

\begin{figure}[h]
  \centering
  \includegraphics{figures/kernel}
  \caption{Classic text-book view of niche differentiation.  Along the
    x-axis is a resource to be partitioned, such as food or prey size,
    or alternatively a trait axis.  Lines represent hypothetical
    resource utilisation curves for five different species, indicating
    the propensity of that species to use resources of a given size.
    The intensity of competition between two species is indicated by
    the degree of overlap in their resource utilisation
    curves. \citet{MacArthur-1967} proposed that competition would
    limit how closely species could sit relative to one another, via
    the width of the utilisation curve. Soon after it's publication,
    this idea of `limiting similarity` appeared in ecology textbooks,
    with similar figures to the one shown appearing in the earliest
    and also the most recent editions of leading textbooks
    \citep{Begon-1986, Begon-2006, Krebs-1978, Krebs-2013,
      Ricklefs-1973, Ricklefs-1999}.}
  % begon-1986: Fig 7.9, pg 263
  % begon-2006: Fig 8.29, pg 257
  % Krebs-1978: Fig 12.2, pg 233
  % Krebs-2013: ??
  % Ricklefs-1973: Fig. 36.13, pg 525
  % Ricklefs-1999: Fig 29.19, pg 602
  \label{fig:competition-kernels}
\end{figure}

\begin{figure}[h]
 \centering
 \includegraphics{figures/components}
 \caption{Method for estimating the shape the competition function embedded
  in a mechanistic resource-based model. To illustrate the method, we present
  results from a model with known solution \citep{Dieckmann-1999} (see Appendix
  \ref{sec:DD99} for details). For each
  model we estimate: (a) the fitness of a rare invader growing without competition,
  (b) the equilibrium population density of invader when growing in monoculture,
  (c1, c2) the fitness of a rare invader growing in competition with an established
  resident. From these elements, we can estimate the shape of the competition
  function (d1, d2). Panels a-b do not vary with the resident community whereas
  panels c-d may.  As intended the method recovers the Gaussian-shaped competition
  functions assumed by this toy model. Moreover, comparison of panels d1 \& d2
  shows that the shape of the function does not vary with resident trait.
  \label{fig:components}}
\end{figure}

\begin{figure}[h]
  \centering
  \includegraphics{figures/rstar}
  \caption{Shape of competition function embedded within the \Rstar model
  for a trait determining the stoichiometry of uptake for two essential resources.
  Here we have only shown the derived  competition functions, see Figs.
  \ref{fig:sm_rstar_components1}-\ref{fig:sm_rstar_components2} for plot of all
  components used to derive these functions.
  Panels a-c show results when consumption of the two resources is symmetric, whereas
  panels d-f show results for an instance when resource 2 is required more than
  resource 1. Panels a \& d show zero-net-growth-isocline (ZNGI) for
  the  model  (solid lines), these are the points at which population growth is zero.
  The series of grey lines are for species differing in the stoichiometry trait.
  The red and blue lines show the ZNGI for a resident species in panels b and
  c, respectively. Panels b-c \& e-f show derived competition functions for a
  community with a single resident at location indicated by the dashed line.
  Solid black lines indicate the competition felt by invading species. Solid
  black bars indicate regions where the invader has positive fitness, i.e.
  could successfully establish. Parameters for the symmetric case, $c_{1i} =
  c_ {2i}=0.5$ and for the asymmetric case $c_{1i} = 0.3, c_ {2i}=0.7$. Resident
  trait values are: (b) 0.6, (c) 0.5, (e) 0.7, (f) 0.9. All other parameters
  are given in Appendix \ref{sec:Rstar}.
  \label{fig:Rstar}}
\end{figure}

\begin{figure}[h]
  \centering
  \includegraphics{figures/plant_hmat}
  \caption{\textbf{Shape of competition function embedded within \plant model for
  the trait height at maturation.}
  Here we have only shown the derived  competition functions, see Fig.
  \ref{fig:sm_plant_hmat_components} for plot of all components used to derive these functions.
  Solid lines show the derived competition functions for a community with a
  single resident, whose location is indicated by the dashed line.
  Resident trait values are: (a) XXX, (b) XXX.
  \label{fig:plant_hmat}}
\end{figure}

\begin{figure}[h]
  \centering
  \includegraphics{figures/plant_lma}
  \caption{\textbf{Shape of competition function embedded within \plant model for
  the trait leaf mass per area.}
  Here we have only shown the derived  competition functions, see Fig.
  \ref{fig:sm_plant_lma_components} for plot of all components used to derive these functions.
  Solid lines show the derived competition functions for a community with a
  single resident, whose location is indicated by the dashed line.
  Resident trait values are: (a) XXX, (b) XXX.
  \label{fig:plant_lma}}
\end{figure}

\begin{figure}[h]
  \centering
  % \includegraphics{figures/plant_lma}
  \caption{\textbf{Density dependence
  ...}.
  \label{fig:dd}}
\end{figure}

\clearpage

\bibliographystyle{amnat}
\bibliography{references}

\section{Appendices}

\subsection{Details about generic Lotka-Volterra model with
Gaussian competition}\label{sec:DD99}

\citet{Dieckmann-1999} analyse a system following the Lotka-Volterra dynamics
described in equation eq. \ref{eq:LVi}, where maximum growth rate is constant
across species, carrying capacity is a Gaussian
function of the invader trait, and the competition function is a Gaussian
based on the difference between traits of the invader and resident:
\begin{subequations}
\begin{align}
\label{eq:DD}
r(x_I) &= r_0, \\
K(x_I) &= K_0 \exp\left(-\frac{\left(x_I-x_0\right)^2}{2\sigma^2_K}\right),\\
\alpha(x_R, x_I) &= \exp\left(-\frac{\left(x_I-x_R\right)^2}{2\sigma^2_C}\right).
\end{align}
\end{subequations}

For the analyses presented in Figure \ref{fig:components} we let $r_0=1.0$, $K_0=500$,
 $s2_C=1$, and $s2_C=0.16$.

\subsection{Details about \Rstar model}\label{sec:Rstar}

The version of the \Rstar model here generally follows
\citet{Huisman-2001}, with the parameter trade-offs assumed by
\citet{Fox-2008}.
%
We consider an unstructured system where individuals compete for two
resources. Let $N_i$ be the abundance of species-$i$ and $R_1, R_2$ be
the abundances of the two resources.
%
The change in population size of species $i$ ($N_i$) is:
\begin{equation}
  \label{eq:rstar-n}
  \frac{\ud N_i}{\ud t} = N_i \, (\min(p_{1i}(R_1), p_{2i}(R_2)) - m)
\end{equation}
where $m$ is the mortality rate and where $p_{ji}(R_j)$ is the
specific growth rate of species $i$ given the availability of resource
$j$, $R_j$, and which follows the Monod funtion:
\begin{equation}
  \label{eq:rstar-p}
  p_{ji}(R_j) = \frac{r_i R_j}{U_{ji} + R_j}.
\end{equation}
%
The parameter $U_{ji}$ represents the efficiency of conversion of
resource $j$ into new offspring; the higher $U_{ji}$, the less
resources need to be acquired for a given growth rate.  (Note that $U$
here is $K$ in \citealt{Huisman-2001} but we avoid $K$ to prevent
confusion with carrying capacity in the Lotka--Volterra model.)
%
The dynamics of the resources themselves follow the equations:
\begin{equation}
  \label{eq:rstar-r}
  \frac{\ud R_j}{\ud t} =
  D \times (S_j - R_j) - \sum_{i=1}^n {C_{ji} N_i
    \min(p_{1i}(R_1), p_{2i}(R_2))}
\end{equation}
where $D$ is the resource turnover rate, $S_j$ is the supply rate of
resource $j$ and where the sum is over all species in the community.
% TODO: Do we have a better definition of this?
The term $C_{ji}$ represents the rate of consumption of resouce $j$.

% NOTE: is there anything else about the model we need to detail?
% The carrying capacity, etc, are all analytic.

\subsection{Details about \plant model}\label{sec:plant}

Excerpt from  \plant model. Hold off writing this until know whether paper describing
the model has been published or not.

\end{document}

%%% Local Variables:
%%% mode: latex
%%% TeX-master: t
%%% TeX-PDF-mode: t
%%% End:

Overall shading caused by communities as an alternative way forward:
if all we're tying to predict is whether or not a species can invade
then the low order approximation that an invader does not affect shade
level is OK and equilvanet to invasion fitness.
