\documentclass[a4paper,11pt]{article}
\usepackage[osf]{mathpazo}
\usepackage{ms/ms}
\usepackage{natbib}
\usepackage{graphicx}
\usepackage{caption}
\usepackage[labelfont=bf]{caption} % make label for figure bold
\usepackage{tabularx}
\usepackage{adjustbox}
\usepackage{color}

\doublespacing

% How to use comments:
% - if you want comments and colors to show, make sure that the lines under "COMMENTS & REVISIONS -- ON" DO NOT have a '%' in front of them, but the ones after "COMMENTS & REVISIONS -- OFF "
% - to remove colors and comments, switch the '%% to the first block (and make sure the second block does not have it).

%%% COMMENTS & REVISIONS -- ON %%%%
\newcommand{\rev}[1]{{\color{navy}{#1}}}
\newcommand{\verify}[1]{{\color{navy}{(verify: #1)}}}
\newcommand{\todo}[1]{{\color{navy}{(todo: #1)}}}


%%% COMMENTS & REVISIONS -- OFF %%%
% \newcommand{\rev}[1]{{#1}}
% \newcommand{\verify}[1]{{#1}}
% \newcommand{\todo}[1]{}

% Instructions for Am Nat: https://www.journals.uchicago.edu/journals/an/instruct?mobileUi=0&


%---------------------------------------

\title{ARTICLE: Emergent shapes of trait-based competition functions from resource-based models: A Gaussian is not normal in plant communities}

\author{Daniel S. Falster\textsuperscript{1,*}, Georges Kunstler\textsuperscript{2}, Richard G. FitzJohn\textsuperscript{3}, Mark Westoby\textsuperscript{4}}

\affiliation{
	\textsuperscript{1} Evolution \& Ecology Research Centre, and School of Biological, Earth and Environmental Sciences, University of New South Wales, Sydney NSW 2052, Australia\\
	\textsuperscript{2} Univ. Grenoble Alpes, INRAE, LESSEM, 2 rue de la Papeterie-BP 76, St-Martin-d’Héres, F-38402 France\\
	\textsuperscript{3} Department of Infectious Disease Epidemiology, Imperial College London, Faculty of Medicine, Norfolk Place, London W2 1PG, United Kingdom\\
	\textsuperscript{4} Department of Biological Sciences, Macquarie University, Sydney NSW 2109 Australia;
	\textsuperscript{*} Author for correspondence: daniel.falster@unsw.edu.au
	}

\runninghead{Competition functions in resource models}
\keywords{Competition, resource, coexistence,}

\newcommand{\ud}{\ensuremath{\mathrm{d}}}
\newcommand{\sign}{\mathop{\mathrm{sign}}\nolimits}
\newcommand{\Rstar}{\ensuremath{R^*}}
\newcommand{\plant}{{\tt plant}}
\newcommand{\hmat}{\ensuremath{h_{\text{mat}}}}
\newcommand{\lma}{\textsc{lma}}

\begin{document}

% Disable words breaking over lines for final submission:
% \raggedright
% \pagestyle{empty}

\mstitlepage
\parindent=1.5em
\addtolength{\parskip}{.3em}

% \doublespacing

\newpage 

\begin{abstract}


In community ecology, it is widely assumed that organisms with similar traits compete more intensely with one another for resources. This assumption is often encoded into theory and empirical tests, via a unimodal competition function, which predicts that per capita competitive effect declines with separation in traits. Yet, it remains unknown how well this function represents the true effect of traits on competitive outcomes, especially for long-lived plant communities where lifetime fitness is difficult to estimate. Here we evaluate the shape of competition functions embedded in two resource-based (RB) models, wherein plants compete for shared, essential resources. In the first RB model, individuals compete for two essential nutrients, and in the second they compete for light in a size-based successional setting. We compared the shapes of the competition functions that emerged from interactions within these RB models to the unimodal function and others shapes commonly applied. In few instances did the trait-based competition function emerging from the RB model even vaguely resemble any of the shapes previously used. The mismatch between these two approaches suggests that theory derived using fixed competition functions based on trait separation may not apply well to plant systems where individuals compete for shared resources. The more promising path will be to model depletion of resources by populations in relation to their traits, with its consequences for fitness landscapes and competitive exclusion. 

\end{abstract}

\section{Introduction}

Competition is a powerful force structuring ecological communities \citep{Schluter-2000, Vellend-2016, Edwards-2018}. In the short term, competition for scarce resources regulates the abundances of different species within a community. Over longer time frames, competition can reshape community structure and dynamics, by favouring some strategies over others. Moreover, these selective effects can drive evolutionary arms races, which favour traits giving prior access to a resource, while also regulating the diversity of strategies that can coexist \citep[e.g.][]{Gause-1934, Lack-1947, MacArthur-1967, Chesson-2000, Falster-2017}. Understanding how traits influence the dynamics of competitive interaction among individuals is thus a foundation for theories of community assembly and adaptive evolution.

A common idea is that organisms with similar trait-values compete more intensely with one another for resources. This idea is expressed via unimodal (often Gaussian) ``competition functions'' (Table \ref{tab:shapes}), that describe the per capita competitive impact of individuals in one population on the lifetime fitness of individuals in another as a function of the difference in their traits. The Gaussian function expresses that the strength of competitive impact declines with the absolute difference in the traits of the two strategies. The idea that competitive intensity declines with trait difference traces back at least 40 years to \citet{MacArthur-1967}’s theory of limiting similarity, which envisaged coexisting species as being distributed along a resource utilisation spectrum, with each species focussing on a different part of the spectrum (Fig. \ref{fig:competition-functions}). This image of species arranged along a resource axis and differentiated in their resource use via some trait has appeared within community ecology textbooks ever since, as an archetypal model of niche differentiation \citep[e.g.,][]{Krebs-1972, Ricklefs-1973, Ricklefs-1999, Krebs-2013}. It was originally derived with reference to specific resource utilisation curves \citep{MacArthur-1967, Roughgarden-1979}. But in modern ecological theory, this type of niche differentiation is usually achieved by embedding a trait--based unimodal (most often Gaussian) competition function within the phenomenological Lotka-Volterra family of population models \citep{Taper-1985, Case-2000, Dieckmann-1999, Leimar-2013, DAndrea-2019}. A unimodal function is also implied in a variety of empirical analyses, including testing for trait dispersion as a signature of niche differentiation \citep{Kraft-2008, Adler-2013, Maire-2012}, methods estimating how competition influences long-term macroevolution \citep{Schluter-2000a, Harmon-2019, 
Fraser-2020a}, and, is also implied by theories of diversity that encode the idea that competition within species is stronger than competition between species \citep{Chesson-2000, Scheffer-2006}. Establishing the evidence base for assumed competition functions is therefore essential to progressing community ecology.

Despite widespread use of a unimodal or Gaussian function, a fundamental question remains: does the competitive impact of an individual in one species on lifetime fitness of individuals in another species decline with a difference in traits? A unimodal function may indeed be appropriate when modelling competition among individuals in animal populations, as differences in traits often do lead to those populations using separately-depletable resources (as envisioned in Figure 1). Classic examples include beak, gape, and body size in vertebrates \citep{Schluter-1984, Schluter-2000a}. However, in plants, differences in traits among species or populations rarely leads to them depleting different resources, as all require the same few basic resources of light, water, and nutrients. Further, competition for these resources is often thought to be asymmetric, such that individuals with traits that give prior access to a resource have a disproportionate effect on all other individuals \citep{Law-1997}. For these situations an asymmetric competitive function, where strategies at one end of the trait spectrum suppress the other end much more than the reverse (Table \ref{tab:shapes}), may capture the dynamics of resource competition better than a unimodal function.

Unfortunately, the biology of plant communities makes it near impossible to measure how any single trait affects lifetime competitive impact. Lifespans extend up to centuries and recruitment is often episodic so that age distributions are not stable. Measuring lifetime fitness for these species is not possible. Moreover, communities often include many species with a wide range of values for each trait. This makes it difficult to isolate the impact of one trait from another. Even in communities of annual plants, it has proven difficult to manipulate individuals so that the competitive impact of a specific trait can be isolated over a sufficient temporal and spatial scale \citep{Kraft-2015}. Some studies have captured the influence of trait values on competitive impact on instantaneous growth or mortality rate \citep{Uriarte-2010, Kunstler-2012, Kunstler-2016, Fort-2014}, but this is not equivalent to lifetime fitness, and thus far data exists on how traits affect competition integrated across the lifecycle \citep{Laughlin-2020}.

In the absence of direct empirical data, mechanistic models offer a promising pathway to assess what real competition functions look like. A strength of models is that they integrate the outcomes of many interacting processes, over long timeframes, large populations, and large spatial scales. For plant communities, competition is primarily mediated through the few essential resources used by all species. Resource-Based (RB) models are a type of process-based model where populations and resources are models explicitly, based on known mechanisms for resource uptake and renewal. Importantly, the lifetime competitive impact of one strategy on another in RB models is not specified directly, but rather emerges via the cumulative effect of competition for the shared resources across an individual's lifetime. By explicitly modeling the depletion of resources by populations of competing individuals, RB models offer a mechanistic link between competition and traits in plant communities. 

In this paper, we test whether the signature of competitive interactions that emerges from RB models of plant communities resembles the trait-dependent competition functions often assumed within community ecology, in a unimodal or Gaussian function. As commonly applied in theoretical ecology, competition functions involve several strong assumptions about the dynamics of competitive interactions. Specifically, they assume that all competitive interactions between two species can be summarised via a single constant competition coefficient, that these values can be represented via a function with a definite shape (such as in Table \ref{tab:shapes}), and that the shape of this function depends only on the difference in traits among species, and not on the absolute value of the trait. We ask how well these assumptions are supported, and critically, whether the shapes of the competition functions that emerge from the RB models, where the abundance and depletion of resources is represented explicitly, support any of the commonly assumed shapes? 

We investigate these questions using two different RB models that have been applied to plant communities and offer different levels of detail and potential links to real systems. One is Tilman's consumer-resource (or ``\Rstar'') model \citep{Tilman-1977, Tilman-1982}, which focuses on the minimal resource level individuals can tolerate. The consumer-resource model is one of the best studied and cited models of resource competition \citep{Tilman-1977, Tilman-1982, Huisman-2001}, originating from a study on exploitative competition for any pair of generic resources \citep{Leon-1975}. While often configured to handle a large number of potential substitutable resources \citep[e.g.][]{Abrams-2008}, we use a version configured for just two essential resources. This seems most relevant for plant communities, which compete for just a few common resources (light, water, nutrients). While it does represent competition for resources, a limitation of this model is that there is no direct link between the parameters of the model, and any of the key plant traits that are currently being measured at large scale \citep{Kattge-2011}. Because of this, the model has to be calibrated for each species and can not provide general predictions of how plant traits connect to competition directly relevant to field studies. 

The second RB model is an example of a size-based successional forest model. Successional models include size-structured competition for light \citep{Huston-1987, Kohyama-1993,Moorcroft-2001, Falster-2011, Falster-2016,Weng-2019}, and potentially also competition for nutrients or water \citep{Moorcroft-2001,Farrior-2013, Smith-2014, Fisher-2018,Weng-2019}. Successional models often have a higher level of detail, capturing individual trajectories across their lifecycle. Importantly, the physiological traits that can be handled by successional models include properties of leaves, stems and reproduction that are commonly measured in the field and known to differentiate function across species. Successional RB models thus offer a unique opportunity to evaluate the suitability of various trait-based competition functions that are directly relevant to field studies of long lived plant communities.

\section{Methods}

To estimate the shape of the emergent competition functions implied in these two RB models, we approximated the behaviour of the RB model with a classic phenomenological trait-based Lotka-Volterra model. To do so we computed all the parameters required by the Lotka-Volterra model except the competition function. This enabled us to estimate the shape of the emergent competition functions implied by the RB models.

\subsection{Standard Lotka-Volterra model}

In the standard Lotka-Volterra model, the rate of increase in the abundance of population $i$ is expressed as a function of the abundance of population $i$, $N_i$; the intrinsic maximum possible per-capita growth rate of population $i$, $r_i$; and the degree to which that maximum is reduced by competition with individuals of the same and other population. A commonly used form \citep[e.g.][]{Lotka-1932, Andrewartha-1953, Dieckmann-1999} is 
\begin{equation} \label{eq:LV} 
	\frac{\ud N_i}{\ud t} = N_i\, r_i \left(1 - \sum_j \frac{\alpha_{ij} N_j}{K_i}\right), 
\end{equation}
where $K_i$ is the equilibrium value for $N_i$ when population-$i$ is grown in a monoculture (i.e., the carrying capacity), and $\alpha_{ij}$ is the per-capita competitive effect of population-$j$ on population-$i$, and the sum is over all population in the community, including population $i$. 

Eq. \ref{eq:LV} implies competitive effects are additive over all individuals in a community and per capita growth declines linearly with the abundance of each population. By convention, competition coefficients are expressed as the effect of population-$j$ on population-$i$, relative to the effect of population-$i$ on itself, so $\alpha_{ii} \equiv 1$. While subtle variations exist how carrying capacity is included in equation (\ref{eq:LV}) \citep[e.g.][]{Chesson-2000, Leimar-2013}, they represent competitive impact in an equivalent fashion and so produce similar results.

\subsection{Method for estimating emergent competition functions}

We are interested in situations where the different parameters in equation \ref{eq:LV} are functions of traits; that is in expressing the subscripts in the above equations as continuous functions of a population’s traits $x$. In this case the coefficient $\alpha_{ij}$ becomes a function of traits: $\alpha(x_i, x_j)$. Various trait-based competition functions (Table \ref{tab:shapes}) are imposed within eq. \ref{eq:LV} to represent different types of competition. Here, instead of imposing a function of predetermined shape, we rearrange the model and use it to solve for emergent shape implied by the RB model. 

We restricted our analysis to the simplest situation where a single type with trait $x_I$ invades a community comprising a single resident type with trait $x_R$ at equilibrium (i.e., $N_j = K_j$). The fitness $w_I(x_I, x_R)$ of the invading type is its per capita rate of increase when rare ($N_I = 0$), or:
\begin{equation}\label{eq:LVi}
 w_I(x_I, x_R) = \frac{\ud N_I}{\ud t} \frac{1}{N_I} =
 r(x_I)\left(1 - \alpha(x_R, x_I) \frac{K(x_R)}{K(x_I)}\right).
\end{equation}

Several ways of computing competition coefficients have been proposed \citep[e.g.,][]{Abrams-1987,Abrams-2008}, though most are focused on testing the assumption of linearity and constancy of competition parameters and not on the shape of the competition functions in relation to traits. Our estimate was calculated simply by solving for the competition coefficient in the LV-T model. \citet{Ricklefs-1973} suggested a similar approach for estimating the size of competition coefficients from field data.

To estimate the shapes of competition functions embedded within the RB models, we used each model to calculate all terms in eq. \ref{eq:LVi} except $\alpha(x_R, x_I)$:
\begin{enumerate}
\item $r(x_I)$: maximum possible fitness of invader, i.e. fitness in the empty environment,
\item $K(x_I)$: equilibrium population density of invader when growing in monoculture,
\item $K(x_R)$: equilibrium population density of resident when growing in monoculture, and
\item $w_I(x_I, x_R)$: fitness of rare invader with traits $x_I$ in competition with an established resident with traits $x_R$.
\end{enumerate}

For each model, we computed these core quantities across a range of invader trait values. We then solved for $\alpha(x_R, x_I)$ by rearranging eq. \ref{eq:LVi} to estimate the emergent competition functions as
\begin{equation} \label{eq:alpha}
 \alpha(x_R, x_I) = \frac{K(x_I)}{N_R} \left(1 - \frac{w_I(x_I, x_R)}{r(x_I)}\right),
\end{equation}
where $N_R = K(x_R)$ (abundance at equilibrium). 

To test if the estimated shape of $\alpha(x_i, x_j)$ differed with the absolute value of the resident trait, we compared the estimated shapes for two different resident communities. For these we chose one where the resident community is at any evolutionary attractor, i.e. trait value towards which selection (in the model) drives the system, and another arbitrary point away from the attractor. We contrasted these two trait values as the competitive effect is likely to vary when the trait is at or away of the attractor. The evolutionary attractor was obtained by solving for the resident trait where the selection gradient is 0 \citep{Geritz-1998}, i.e. the value of $x_R$ giving
\begin{equation} \label{eq:equil}\frac{\delta w_I(x_I, x_R)}{\delta x_I} |_{x_I = x_R} =0.
\end{equation}

\subsection{Methodological test: A model with known competition function}

To verify that our approach for estimating competition functions accurately recovers the competition functions used within a model, we tested it on a model with a known solution. Using the model of \citet{Dieckmann-1999}, we estimated the shape of the competition function embedded within the equations, as if we didn't know the answer, and compared this to the true generating value.

The model from \citet{Dieckmann-1999} considers a system following the dynamics described in equation eq. \ref{eq:LVi}, where maximum growth rate is constant across populations, carrying capacity is a Gaussian function of the invader trait, and the competition function is a Gaussian, based on the difference between traits of the invader and resident:
\begin{subequations}
\begin{align}
\label{eq:DD}
r(x_I) &= r_0, \\
K(x_I) &= K_0 \exp\left(-\frac{\left(x_I-x_0\right)^2}{2\sigma^2_K}\right), \textrm{and}\\
\alpha(x_R, x_I) &= \exp\left(-\frac{\left(x_I-x_R\right)^2}{2\sigma^2_C}\right).
\end{align}
\end{subequations}
Here $x_0$ is the evolutionary attractor trait value, while $\sigma^2_K$ and $\sigma^2_C$ are the variances of the Gaussian functions for carrying capacity and competition function, respectively. Following \citet{Dieckmann-1999}, we let $r_0=1.0$, $x_0=0.0$, $K_0=500$, $\sigma^2_K=1$, and $\sigma^2_C=0.4$ -- a scenario which enables invasion of new strategies at the evolutionary attractor \citep{Dieckmann-1999}.

Our proposed method for recovering the shape of a competition function within a model was able to accurately recover the Gaussian-shaped competition function used by the model of \citet{Dieckmann-1999}. Fig. \ref{fig:components}a-f show how elements of the model vary for two different resident communities: one away from (\ref{fig:components}c-d) and one at (\ref{fig:components}e-f) the evolutionary attractor. In this model, selection drives traits towards a point where the fitness landscape becomes a minimum, and therefore evolutionary branching could potentially occur \citet{Dieckmann-1999}. The resulting competition functions estimated by our method are shown in Fig. \ref{fig:components}d and Fig. \ref{fig:components}f, respectively. Both estimated functions recovered the true Gaussian function applied within the original model. Note also that the competition functions do not vary in shape, only in position (compare panels d and f) along the trait axis for the different resident scenarios. This illustrates that the functions depend only on the difference in trait values, not the absolute value of trait values.

\subsection{RB model 1: Consumer-resource model}

Much like real plants, individuals in the consumer-resource model compete for essential resources. The environment is unstructured, so success is based on rate of resource uptake and conversion into biomass. We implemented a version where the growth rates follow a Monod equation \citep{Huisman-2001}, with two resources and trade-offs assumed by \citet{Fox-2008}. Per capita growth rates for each strategy depend on the availability of the resource that is currently most limiting. Let $R_1, R_2$ be the abundances of the two resources. The change in population size of type $i$ is then
\begin{equation}
 \label{eq:rstar-n}
 \frac{\ud N_i}{\ud t} = N_i \, \left(\min\left(\frac{r_i R_1}{U_{1i} + R_1}, \frac{r_i R_2}{U_{2i} + R_2}\right) - m\right),
\end{equation}
where $r_i$ is the maximum growth rate of type $i$; $m$ is the mortality rate, $U_{ki}$ is the half-saturation of resource $k$ for type $i$ (higher $U_{ki}$ means more resources are needed for a given growth rate). The dynamics of the resources themselves are given by
\begin{equation}
 \label{eq:rstar-r}
 \frac{\ud R_k}{\ud t} =
 D (S_k - R_k) - \sum_{i=1}^n {C_{ki} N_i
 \min\left(\frac{r_i R_1}{U_{1i} + R_1}, \frac{r_i R_2}{U_{2i} + R_2}\right)},
\end{equation}
where $D$ is the resource turnover rate, $S_k$ is the supply rate of resource $k$, 
$C_{ki}$ is the content of resource $k$ by type $i$.

Following previous studies \citep{Tilman-1985, Schreiber-2003, Fox-2008}, we let a trait $x_i$ determine the relative amounts of the two resources required for growth of individuals in type $i$: \begin{equation}
\label{eq:R3}
x = U_{1i} = 1 - U_{2i}.
\end{equation}
Populations thus vary from requiring mostly resource 2 ($x << 0.5$), to equal amounts of both resources ($x = 0.5$), to mostly resource 1 ($x >> 0.5$).

We considered two cases of the model for resource content. First, we considered a model where resource contents are constant for the two competing strategies. Because we assume that all strategies have identical resource contents ($C_{.i} = C_{.j}$), coexistence was not possible in this version \citep{Tilman-1982}. Invasion is therefore sufficient for competitive replacement. In this scenario, we explored two parameterizations: a) symmetric resource contents, with $C_{.i} = C_{.j} = [2,2]$ for both strategies and a supply rate of $S = [0.5, 0.5]$, and b) asymmetric resource content, assuming $C_{.i} = C_{.j} = [0.3,0.7]$ and $S = [0.5, 0.5]$. Second, we considered a case where the resource content of each strategy was equal to its half-saturation for that resource $U_{ki} = C_{ki}$. Under this assumption, each strategy has a higher resource content for its most limiting resource, which is a key ingredient for coexistence \citep{Tilman-1982}. In this second scenario, we again explored two parametrisations: a) symmetric resource supply, with $S = [0.5, 0.5]$, and b) asymmetric resource supply, with $S = [0.7, 0.3]$. The other parameters were set as $m = 0.25$, $D=0.25$, and $r_i = 1$ for all analyses. 

\subsection{RB model 2: Size-based successional dynamics}

\plant\ is an individual-based model of plant competition for light in a size-structured metapopulation \citep{Falster-2016, Falster-2017}. The \plant\ model is used here as an example of a broader group of models dealing with size-based successional turnover. Several forest models have a similar overall ecological structure, with the forest comprising a metapopulation of patches \citep{Huston-1987,Kohyama-1993,Moorcroft-2001, Medvigy-2012, Smith-2014}. A key feature of \plant, for current purposes, is that individual performance and resource use are based on well-established physiological tradeoffs represented by traits widely measured in the field. Because of this, the competition functions retrieved are directly relevant to field data.

Vegetation in the \plant\ model comprises a large number of patches subject to a disturbance regime. Disturbances clear established vegetation within a patch, which are then recolonised via seed dispersal from other patches. Individual plants progress through their life-cycle, from seeds until reproductive adults. Growth, survival and fecundity rates are determined by the combined effect of an individual's traits, their current size, and the degree of shading from other plants within the same patch. Shading at a given height is computed from the cumulative leaf area above that height. Short-term competitive effects are therefore size-asymmetric. The sole effect of shading (and the only source of competition) is to alter an individual's photosynthetic income. Following \citet{Falster-2017}, fitness of individuals with a given type or trait-combination is calculated by integrating seed output across the entire metapopulation. 

Using \plant, we estimated trait-based competitive functions for two traits whose evolutionary dynamics have recently been investigated \citep{Falster-2017}: leaf mass per unit leaf area (\lma) and height at maturation (\hmat). In the model, \lma\ moderates a physiological trade-off between the cost of building an extra unit of leaf area and the rate of leaf turnover \citep{Wright-2004}. Variation in \lma\ thereby underpins a trade-off between height growth rate and shade-tolerance of seedlings. Meanwhile, \hmat\ moderates the allocation of surplus energy between growth and seed production, as an individual increases in size \citep{Falster-2017}. Full details are in \citet{Falster-2016, Falster-2017}.

\section{Results}

\subsection{Consumer-resource model}

For the consumer-resource model, we explored the competition function relating to a trait influencing the relative amounts of the two resources required for growth of individuals under two cases (see methods for details): a case with constant resource content which precludes coexistence and a case where resource content is equal to the resource requirement, which allows coexistence. For each of these cases we explored different parameterizations to assess the robustness of our results.

A notable result that emerged from these different models was that the shape of the emergent competition function relative to the trait of the resident varied drastically with the absolute location of the resident (competition function is offset from the resident trait value when the resident is away of the evolutionary attractor in the Figure 3, Figures S1, S2, and S3 in Supplementary Materials). This contradicts with the classical unimodal competition functions, such as the Gaussian, which are always centred on the resident trait value. 

A second result was that the shape of the emergent competition function varied drastically depending on whether or not we constrained the resource content to equal requirements. When the model structure allowed coexistence, such as in the figure 3, the emergent competition function was unimodal, resembling some of the kinked competition functions in Table \ref{tab:shapes}. However, the function was only centred on the resident trait at the evolutionary attractor when $x = 0.5$, being offset when $x = 0.6$. In contrast, when the structure of the model did not allow coexistence (constant resource content), the shape of the emerging competition function was inverted, suggesting that the per capita effect of the resident is stronger with individuals further away from the evolutionary attractor (Figure S1). In addition, when considering either asymmetric resource content in the case one, or asymmetric resource supply in the case two, the competition functions were considerably more complicated (respectively Figure S2 and S3). 

\subsection{Size-based successional dynamics: \hmat}

In the \plant\ model, strategies that have different heights at reproductive maturity \hmat\ differ in how they allocate resources between growth and seed production at different sizes. Plants with lower \hmat\ allocate effort to seed production earlier, increasing the chances they will achieve some reproduction, but because from that point on they allocate less resources to growth, they are liable to be over-topped and out-shaded by strategies that reproduce later \citep{Falster-2017}.

Reflecting this dynamic, the competition functions that emerged from size-based successional dynamics for the trait \hmat\ were asymmetric (Fig. \ref{fig:plant_hmat}d, f), somewhat resembling the asymmetric function used in some previous studies \citep{Law-1997, Kisdi-1999, Geritz-1999, Egas-2004, Calcagno-2006, DAndrea-2013} (Table \ref{tab:shapes}). This shape arose from joint changes in maximum growth rate, carrying capacity, and fitness with trait values (Fig. \ref{fig:plant_hmat}). As expected, all strategies exerted competition more strongly on strategies that are shorter than them, and less on strategies that are taller than them. However, the competition function did not asymptote like the logistic function that has previously been used to model asymmetric competition (Table \ref{tab:shapes}). The shape of the function also varied with the position of the resident type (compare panels d and f in Fig. \ref{fig:plant_hmat}). As the resident strategy increased in size to reach the evolutionary attractor, the apparent per capita competition was higher (Fig. \ref{fig:plant_hmat}d,f), yet the fitness of smaller strategies improved (Fig. \ref{fig:plant_hmat}c, e).

\subsection{Size-based successional dynamics: \lma}

Strategies with different \lma\ values reflect different compromises struck along the leaf economic spectrum \citep{Reich-1997,Wright-2004}. High \lma\ leaves are costly to deploy but have longer lifespan, so need to be replaced less often. Previous analyses have shown that in a single population system, selection initially favours lower \lma\ values, as the fast growth achieved gives prior access to light \citep{Falster-2017}. On this basis, we might expect the competition function for this trait to also have an asymmetric shape.

The shape of the competition function from size-based successional dynamics for the trait \lma\ was complex (Fig. \ref{fig:plant_lma}d) and does not conform to any of the simple shapes in common use (Table \ref{tab:shapes}). This shape arose from joint changes in maximum growth rate, carrying capacity, and fitness with trait values (Fig. \ref{fig:plant_lma}). In contrast to the complex shape of fitness and competition, maximum growth rate and carrying capacity had simple forms, with carrying capacity being approximately Gaussian (Fig. \ref{fig:plant_lma}\,a and b). However, small undulations in the fitness landscapes (Fig. \ref{fig:plant_lma}c, e) were amplified when expressed as a competition function (Fig. \ref{fig:plant_lma}d, f). 

We first looked at the situation where a resident was away from the evolutionary attractor, under directional selection towards lower \lma\ (Fig. \ref{fig:plant_lma}c). In the region of trait space close to the resident type, the emergent competition function was strongly asymmetric (Fig. \ref{fig:plant_lma}d). The resident exerted a large amount of competition on strategies with \lma\ values higher than itself (``slow''strategy strategies) and less on strategies with \lma\ values lower (``fast'' strategy strategies; Fig. \ref{fig:plant_lma}c). Invasion was possible in the region with \lma\ values smaller than the invader, until growth and carrying capacity reached a biological limit to invasion. Above the resident \lma\ value, the competition function was sawtooth shaped, before eventually declining steeply.

Moving the resident towards the evolutionary attractor (Fig. \ref{fig:plant_lma}e), the emergent competition function (Fig. \ref{fig:plant_lma}f) differed in shape to when the resident was away from the attractor, indicating the shape of the function depends on the location of the resident, as well on difference in trait values. Invasion became possible through a broad range of trait values, including just below and above the resident value. 

\section{Discussion}

RB models offer an exciting opportunity to study how differences in traits relate to the competitive impact of one strategy on the fitness of another, particularly in long lived plant communities where direct empirical measurement of lifetime fitness is rarely possible. These competitive impacts underpin theories of species coexistence, trait evolution, and response to ecological communities to changing climate \citep{ Laughlin-2020}. We inferred the shape of the trait-based competition functions that emerge from two different RB models, in which individuals impact one another via depletion of one or more essential resources. The competition functions that emerged from these RB models bore little resemblance to the collection of trait-based competition functions shapes used in community ecology, including the unimodal Gaussian function. Our results thereby suggest that a common assumption, found in both theoretical and empirical studies, that individuals compete more intensely when they have similar traits, has limited support in plant communities. Some of the competition functions recovered resembled other functions used in plant community ecology, such as the asymmetric function, where strategies have a one-sided impact. Yet, the shape of all the emergent functions recovered also varied with the identity of the resident, not just the difference in trait values, as is commonly assumed. In other words, the RB models did not present a single fixed shape for the competition functions. Combined, our suggest that plant community ecology should not rely on theory using trait-based competition functions, as these fail to adequately capture the effect of traits on fitness. We should instead shift to models where resource depletion is modelled directly. 

\subsection{Competition functions for plants are not unimodal nor even Gaussian}

Trait-based competition functions are difficult to estimate in the field, and this is one of the reasons we used models to study their shape. Originally, Lotka-Volterra models were formulated using pairwise competition coefficients, calculated between pairs of species. A substantial literature exists around measurement and interpretation of such coefficients \citep[e.g.][]{Ricklefs-1973, Abrams-1980, Law-1997a,Freckleton-2001, Kraft-2015}. One approach involves measuring the different elements of eq. \ref{eq:LVi} then using these to estimate the competitive interaction (Fig. \ref{fig:components}) \citep{Ricklefs-1973}. This is the method we applied here, only using models instead of real organisms. But for plants, estimating either lifetime fitness or pairwise competition coefficients is typically only possible for short-lived species \citep{Kraft-2015, Mayfield-2017, Law-1997,Freckleton-2001,Laughlin-2020}. 

Moreover, field and experimental data are only ever going to give point estimates of the pairwise competitive effect between two species ($\alpha_{ij}$), not a continuous, trait-based function. This is a much bigger and probably intractable challenge. Estimating such a function would require fitness to be measured across the life cycle for a large number of strategies varying only in the focal trait. 

With limited hope of ever measuring the shape of competition functions from empirical data, RB models offer a tool for honing our understanding. Models allow us to integrate empirical data and run experiments to compare across hypothetical strategies in ways that are simply not possible in real systems. Ideally, models integrate existing empirical and conceptual understanding, and then point the way forward.

It is striking that the competition functions derived from RB models at best only vaguely resemble the dominant shapes used in most theoretical investigations. The most common forms used have included the Gaussian, asymmetric (offset) Gaussian, and asymmetric shapes (Table \ref{tab:shapes}). The competition functions from the consumer-resource model resembled the Laplacian kernel, but only when it was configured so that resource content varied with uptake, not when the resource content was constant. The derived function from the \hmat\ within the size-based successional model mildly resembled the asymmetric functions used in models investigating asymmetric competition \citep{Law-1997, Geritz-1999, Kisdi-1999, Calcagno-2006}. Yet even then, the resemblance was superficial only; and the shape of the emergent function varied with the identity of the competitor. The competition function for our \lma\ model has hints of an asymmetric Gaussian, but the ridges along the top are distinctly non-Gaussian and arise from real biological features of the system. \citet{Falster-2011} showed how the amount of shading undulates through successional time in the meta-community; here this translates into undulations in the fitness of an invader with respect to traits.

Our results therefore imply the convenient assumption that competition functions can be defined by a simple function based solely on differences in trait values -- without reference to population size, or absolute value of traits, or resource levels -- may not effectively capture RB competition in plant communities. While our analysis has focussed on plants competing for essential resources, similar concerns have been raised about competition functions for systems when individuals compete for substitutable resources \citep{Abrams-2008,Song-2019}. These concerns add to those expressed previously regarding competition coefficients and competition functions. \citet{Abrams-1980} argued that in the real world there might be widespread density dependence in the competition function - i.e. the $\alpha_{ij}$s would vary with density. Recently, both empirical and analytical approaches have also shown that in multi species systems competition coefficients are not additive across species, as they are assumed to be \citep{Mayfield-2017,Levine-2017, Letten-2019}. In the same vein adding higher order terms accounting for this nonlinearity improves the predictive performance of models \citep{Letten-2019}. Combined, these results suggest that trait-based competition functions with the shapes typically assumed will be a poor guide to the outcomes of competition in plants.

\subsection{Where did competition functions come from?}

Given the difficulty of measuring trait-based competition coefficients, and their lack of alignment with RB models, it is worth revisiting their origins, to see what underpinned their emergence into the literature.

In 1960s theory and 1970s textbooks, niches were typically thought of as utilization functions along a continuous resource axis \citep{MacArthur-1967} (Fig. \ref{fig:competition-functions}). The resource in question would be depletable or pre-emptable, in order for overlap of utilization functions to translate into competition. If a trait was used as a niche index, it was chosen on grounds of mapping tidily on to a depletable resource, such as beak size for seed-eating finches. As a first approximation, MacArthur \citet{MacArthur-1967, MacArthur-1970} showed that a Lotka-Volterra type model with constant pairwise competition coefficients (i.e. the $\alpha_{ij}$ parameters) could be derived from the degree of overlap between utilization functions between two species \citep{MacArthur-1967, May-1972, Abrams-1975}. Although it was noted at the time that the Lotka-Volterra model was only a first-order linear approximation to population dynamics, and it was subsequently shown that relaxing assumptions in the derivation brought into question both the constancy and additivity of the competition coefficients \citep{Abrams-1980, Letten-2019}, the idea of a constant competition coefficient has persisted.

Beginning in the 1980s, a desire to model character displacement saw the concept of a utilisation function first approximated via a trait -- the mean of the utilisation function \citep{Roughgarden-1979} -- and second replaced entirely by a competition function based on trait differences \citep{Slatkin-1980,Taper-1985}. This shift enabled models for character displacement \citep[e.g.,][]{Taper-1985, Case-2000, Goldberg-2006} and trait evolution \citep{Brown-1987, Geritz-1998, Geritz-1999, Dieckmann-1999, Abrams-2001} to be developed, as the competition functions allowed pairwise coefficients to be calculated for any hypothetical combination of traits. Gaussian competition functions have been the most-common shape, but other shapes have also been considered (Table \ref{tab:shapes}). An asymmetric function has been used where size gives exclusive access to a resource \citep[e.g.][]{Law-1997, Kisdi-1999, Geritz-1999, Egas-2004, Calcagno-2006, DAndrea-2013} (Table \ref{tab:shapes}). More recently, researchers have considered so-called ``kinked'' functions (Table \ref{tab:shapes}), with shapes that are not continuously differentiable \citep{Calcagno-2006, Scheffer-2006, Pigolotti-2007, Leimar-2013, Barabas-2013}. 

It is now abundantly clear that in Lotka-Volterra style models for communities, the shape of any trait-based competition function fundamentally alters the diversity of strategies supported. Specifically, whether the community will be constituted of a single type, many evenly spaced strategies (limiting similarity), or a continuity of strategies (no limiting similarity), can depend on slight variations in the shape of the competition function \citep{Calcagno-2006, Scheffer-2006, Pigolotti-2007, Barabas-2012,Barabas-2013,DAndrea-2013,Leimar-2013}. Kinked functions (Table \ref{tab:shapes}) are more likely to lead to continuous coexistence, as invader fitness rises steeply with initial trait separation from the existing resident. However, continuous coexistence may also occur with the classical Gaussian function, depending on how traits affect carrying capacity \citep{Leimar-2013}. In other words, predicted communities depend on the details of how competition is modelled.

Also clear, is that at least in plants, the shift towards using trait-based competition functions was never supported with empirical data assessing appropriate shapes.

\subsection{A renewed focus on linking traits to resource use}

It is possible in principle to expand the idea of a competition function. The current formulations are density independent and invariant with respect to the absolute value of traits, but in principle functions could be constructed that varied with population densities and with the actual values of species traits \citep{Abrams-2008,Song-2019}. This would enable more complexities to be incorporated. But with these complications, it would be harder to intuit the influence of residents on invasibility, and what configuration of species would be stable. To be robust, new functions would need to be rigorously derived from another detailed model. As well as depending on traits and densities, it would be prudent to expect that the total competitive effect would not sum neatly across species. So the competition would be a function of the entire resident community. But at that point, are we stretching the concept of a competition coefficient beyond its useful limit? We could just as well model a multi-species assemblage with explicit depletion of resources. To that end, we suggest four promising avenues. 

First, there are currently few models that relate traits to fitness via competition for a common resource, without including a phenomenological competition function \citep{DAndrea-2016a}. There are many ecosystem models and simulators that include competition for explicit resources, but these are often designed to track cycles of carbon, water or nutrients \citep{Moorcroft-2001, Medvigy-2012, Smith-2014,Fisher-2018}, more than to study competition among strategies or community assembly \citep{Kohyama-1993, Falster-2017}. These models commonly provide for a few strategies with fixed properties. Retooling such models so that they can also focus on the dynamics of competing strategies, driven solely by differences in traits, could expand the range of possibilities for capturing the effect of traits on competition.

Second, given the difficulty in linking RB models to competition functions, we suggest shifting focus away from the competition functions or coefficients per se, and focussing instead onto the ways competition impacts on fitness landscapes and resources \citep[see also][]{Laughlin-2020}. A notable feature of our results, particularly those from the size-based successional model, was that the impacts of competition on fitness landscapes are more orderly and intuitive than the shapes of the derived competition functions (Figs. \ref{fig:rstar_components_UC1} - \ref{fig:plant_lma}). Fitness landscapes capture the combined effects of all features of a species demography, including competition. Comparing landscapes with and without competition allows one to see how the growth rate for all strategies is reduced by the presence of an existing type, e.g. compare panel a (no competition) vs panels c and e (with competition) in Figs. \ref{fig:rstar_components_UC1} - \ref{fig:plant_lma}). 

Third, the increased certainty that comes from modelling resources explicitly should give us confidence to explore how multivariate trait combinations interact to shape competitive outcomes. Using the same successional model as analysed here, \citet{Falster-2017} already showed how two traits could interact to produce a different strategy mixture than when each trait was analysed independently. Using Gaussian competition functions, \citet{Doebeli-2010} similarly showed that interactions between multiple traits enhanced the possibility for diverse strategies to be supported. 

Finally, we should be looking to derive new macroscopic approximations of the more-detailed RB models. Simple models can be powerful tools, provided they capture the main emergent features of a system \citep{Levin-2006}. Existing competition functions embedded within Lotka-Volterra models may not approximate the RB models used here very well, but other models may.

\subsection{Conclusions}

Despite being difficult to quantify in the field and to represent in models, competition is undoubtedly a powerful force in structuring communities. Our results reinforce previous arguments that if competition is density dependent or non-additive across species \citep{Abrams-1980, Abrams-2008, Letten-2019}, or if the shape of competition varies with species traits and not just trait separation \citep{Abrams-2008, Song-2019}, then a trait-based competition function that produces a fixed shape will not adequately capture the dynamics of real systems. Moreover, the widespread assumption of the unimodal function that competition is most intense among similar species does not work in plant communities. If we want to move trait-based community ecology forward, we need to focus on connecting traits to resources and population dynamics, with models that explicitly model the resources being contested.

\clearpage

\section{Tables}

\begin{table}[h]
 \caption{Example shapes of competition functions used to represent competition in trait space. `Kinked' functions refers to those that are ``non-differentiable at the point of self-competition'' \citep{Barabas-2012}. Equations are presented only to give a general idea of shapes in terms of species trait differences ($\Delta x$); various scaling parameters are common, and specific to different implementations. }

\centering
{\footnotesize
\renewcommand{\arraystretch}{1.5} % more space between rows of table
\def\minifigheight{5ex}
\begin{tabularx}{\textwidth}{lclp{5cm}}
 \hline
 Name & Shape & Representative equation & Example references \\
 \hline
  \textbf{Continuous}\\[1ex]
  %
  Constant &\adjustbox{valign=t}{\includegraphics[height=\minifigheight]{ms/figures/shape/constant.pdf}}& $c$ & \citet{Hubbell-2001} \\
  %
  Gaussian &\adjustbox{valign=t}{\includegraphics[height=\minifigheight]{ms/figures/shape/gaussian}}& $\exp\left(-\frac{(\Delta x)^2}{\sigma^2}\right)$ & \citet{Slatkin-1980, Taper-1985, Dieckmann-1999}\\
  %
  Platykurtic &\adjustbox{valign=t}{\includegraphics[height=\minifigheight]{ms/figures/shape/platykurtic}}& $\exp\left(-\frac{(\Delta x)^4}{\sigma^2}\right)$& \citet{Leimar-2013} \\
  %
  Exponential &\adjustbox{valign=t}{\includegraphics[height=\minifigheight]{ms/figures/shape/exponential}}& $\exp(d x)$ &\citet{Pigolotti-2007}\\
  % Add scaling term on the dx that's important for connecting with asymmetric step where d = infty
  Asymmetric &\adjustbox{valign=t}{\includegraphics[height=\minifigheight]{ms/figures/shape/logistic}} & $\frac{1}{1 + \exp(d \,\Delta x)}$ & \citet{Law-1997, Kisdi-1999, Geritz-1999, Egas-2004, Calcagno-2006, DAndrea-2013}\\
  %
  Offset Gaussian &\adjustbox{valign=t}{\includegraphics[height=\minifigheight]{ms/figures/shape/gaussian_offset}}& $\exp\left(-\frac{(\Delta x - b)^2}{\sigma^2}\right)$ & \citet{Slatkin-1980, Rummel-1985,Brown-1987-66}\\
  %
  Gaussian with hat &\adjustbox{valign=t}{\includegraphics[height=\minifigheight]{ms/figures/shape/gaussian_with_hat}}& 
  % TODO This eqn looks ugly, wonder if we need it?
  {\tiny $c \exp\left(-\frac{(\Delta x)^2}{\sigma_1^2}\right)+(1-c) \exp\left(-\frac{(\Delta x)^2}{\sigma_2^2}\right)$}
  & \citet{Barabas-2013, Leimar-2013} \\
  %
   \textbf{Kinked}\\[1ex]
  Laplacian &\adjustbox{valign=t}{\includegraphics[height=\minifigheight]{ms/figures/shape/laplacian}} & $\exp(-2|x|)$ & \citet{Roughgarden-1972, Leimar-2013} \\
  %
  Asymmetric step
   &\adjustbox{valign=t}{\includegraphics[height=\minifigheight]{ms/figures/shape/step_asymmetric}} & $ \frac{1}{1 + \exp(\infty \, \Delta x)},$ & \citet{Kinzig-1999, Egas-2004, Calcagno-2006, DAndrea-2013} \\
  Gaussian with spike &\adjustbox{valign=t}{\includegraphics[height=\minifigheight]{ms/figures/shape/gaussian_with_spike}} & 
   {\tiny $c \exp\left(-\frac{(\Delta x)^2}{\sigma_1^2}\right)+(1-c) \exp\left(-\frac{(\Delta x)^2}{\infty}\right)$}
   & \citet{Scheffer-2006, Barabas-2013} \\
  \hline
 \end{tabularx}
}
\label{tab:shapes}
\end{table}

\clearpage

\section{Figures}

\begin{figure}[ht]
 \centering
 \includegraphics{ms/figures/kernel}
 \caption{{\bf Classic textbook view of niche differentiation, based on either Gaussian resource utilisation curve or competition functions.} Along the x-axis is a resource to be partitioned, such as food or prey size, or alternatively a trait axis. Lines represent hypothetical resource utilisation curves for five different species, indicating the propensity of that species to use resources of a given particle size. Competition between two species will be proportional to the overlap in their resource utilisation curves. \citet{MacArthur-1967} proposed that competition would limit how closely species could sit relative to one another, via the width of utilisation curves. Soon after its publication, this idea of `limiting similarity` appeared in ecology textbooks. Figures similar to this appeared in the earliest editions of leading textbooks, continuing up to recent editions: \citet[Fig. 7.9]{Begon-1986}; \citet[Fig. 8.29]{Begon-2006}; \citet[Fig. 12.20]{Krebs-1978}; \citet[Fig. 19.15]{ Krebs-2013}; \citet[Fig. 36.13]{Ricklefs-1973}; \citet[Fig. 29.19]{Ricklefs-1999}.} 
 \label{fig:competition-functions}
\end{figure}

\clearpage

\begin{figure}[ht]
 \centering
 \includegraphics[width=1.0\textwidth]{ms/figures/components}
 \caption{{\bf Demonstration of our method for estimating the shape of a competition function embedded in a model with known solution.} For each model analysed, we use the model to calculate all of the terms in the LV-T model (eq. \ref{eq:LVi}), including: (a) the fitness of a rare invader growing without competition ($r(x_I)$), (b) the equilibrium population density of invader and resident when growing in monoculture ($K(x_I)$, $K(x_R)$), (c, e) the fitness of a rare invader growing in competition with an established resident ($w_I(x_I, x_R)$), represented by a point. From these elements, we can estimate the shape of the competition function (d, f). Solid black bars indicating regions where the invader has positive fitness, i.e. could successfully establish. Note that panels (a)-(b) do not depend on the location of the resident community whereas panels (c)-(f) do. Panels (c, d) show a situation where the resident is away from it's local evolutionary attractor, whereas in (e, f) the resident is at its evolutionary attractor. Resident trait values are: (c,d) x= -1.0, (e,f) x = 0.0. All other parameters are given in the Methods and also in the code.}
\label{fig:components}
\end{figure}

\clearpage


\clearpage

\begin{figure}[ht]
 \centering
 \includegraphics[width=1.0\textwidth]{ms/figures/rstar_components_UC_1}
 \caption{{\bf Shape of competition function embedded within the consumer-resource model for a trait determining the relative amounts of the two essential resources required for growth, with resource content equal to resource half saturation determined by the trait such as $C_{ki} = U_{ki}$.} Panel descriptions as in Fig. \ref{fig:components}. Resident trait values are: (c,d) 0.6, (e,f) 0.5. All other parameters are given in the Methods and also in the code.} 
 \label{fig:rstar_components_UC1}
\end{figure}


\clearpage

\begin{figure}[ht]
 \centering
 \includegraphics[width=1.0\textwidth]{ms/figures/plant_hmat_components}
 \caption{{\bf Shape of competition function emerging from size-based successional dynamics for the trait \hmat.} 
	Panel descriptions as in Fig. \ref{fig:components}. Resident trait values are: (c,d) \hmat = 5, \lma = 0.3, (e,f) \hmat = 15.5,\lma = 0.3. All other parameters are given in the code.}
 \label{fig:plant_hmat}
\end{figure}

\clearpage

\begin{figure}[ht]
 \centering
 \includegraphics[width=1.0\textwidth]{ms/figures/plant_lma_components}
 \caption{{\bf Shape of competition function emerging from size-based successional dynamics for the trait \lma.} 
 Panel descriptions as in Fig. \ref{fig:components}. Resident trait values are: (c,d) \lma\ = 0.17, \hmat = 15 (e,f) \lma\ = 0.07, \hmat = 15. All other parameters are given in the code.}
 \label{fig:plant_lma}
\end{figure}

\clearpage

\bibliographystyle{ms/amnat}
\bibliography{ms/references}

\clearpage

\section{Supplementary online material}

\renewcommand{\thefigure}{S\arabic{figure}}
\setcounter{figure}{0}

\begin{figure}[ht]
 \centering
	\includegraphics[width=1.0\textwidth]{ms/figures/rstar_components_U_1}
 \caption{{\bf Shape of competition function embedded within the consumer-resource model for a trait determining the relative amounts of the two essential resources required for growth, with constant symmetric resource content.} Panel descriptions as in Fig. \ref{fig:components}. Parameters for the resource content, $C = \{2, 2\}$. Resident trait values are: (c,d) 0.6, (e,f) 0.5. All other parameters are given in the Methods and also in the code.}
 \label{figS:Rstar_U1}
\end{figure}

\newpage
\begin{figure}[ht]
 \centering
	\includegraphics[width=1.0\textwidth]{ms/figures/rstar_components_U_2}
 \caption{{\bf Shape of competition function embedded within the consumer-resource model for a trait determining the relative amounts of the two essential resources required for growth, with constant asymmetric resource content.} Panel descriptions as in Fig. \ref{fig:components}. Parameters for the resource content, $C = \{0.3, 0.7\}$. Resident trait values are: (c,d) 0.7, (e,f) 0.9. All other parameters are given in the Methods and also in the code.}
 \label{figS:Rstar_U2}
\end{figure}

\newpage
\begin{figure}[ht]
 \centering
 \includegraphics[width=1.0\textwidth]{ms/figures/rstar_components_UC_2}
 \caption{{\bf Shape of competition function embedded within the consumer-resource model for a trait determining the relative amounts of the two essential resources required for growth, with resource content equal to resource half saturation determined by the trait such as $C_{ki} = U_{ki}$ with asymmetrical resource supply $S = \{0.7, 0.3\}$.} Panel descriptions as in Fig. \ref{fig:components}. Resident trait values are: (c,d) 0.9, (e,f) 0.7. All other parameters are given in the Methods and also in the code.} 
 \label{fig:Srstar_components_UC2}
\end{figure}


\end{document}

%%% Local Variables:
%%% mode: latex
%%% TeX-master: t
%%% End:


