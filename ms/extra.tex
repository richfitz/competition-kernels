
Competition among all types in a community than then be summarised via a matrix of pairwise competition coefficients.

The metaphor of species packing has long been central to community ecology.
This idea supposes there is a volume of niche space, and each potential community member has a ``fundamental niche'' within that volume.
The fundamental niche is compressed by competition into a realised niche, as species are packed in alongside each other (Figure \ref{fig:competition-kernels} for one-dimensional version found in many textbooks).
%
The roots of the species packing metaphor trace to the competitive exclusion principle -- whereby if species have identical niches or compete for the same resource units, all but one are competitively excluded \citep{Lotka-1925, Volterra-1926, Gause-1934}.
The competitive exclusion principle was later converted into the question how dissimilar two species niche need to be in their resource utilization in order to coexist \citep{MacArthur-1967, May-1972, Slatkin-1980}.
%
By the 1970s, a new generation of textbooks built their account of community ecology around three alternative ways of getting more species into a community: larger total niche space, more overlap between adjacent niches, or narrower niches for each species \citep{Whittaker-1970,Krebs-1972,Ricklefs-1973,Pianka-1974}. The species packing metaphor has persisted up to the present day as the mainstream approach to community ecology.

In 1960s theory and 1970s textbooks, niches were typically thought of as utilization functions along a continuous resource axis\citep{MacArthur-1967,Abrams-1975} (Fig. \ref{fig:competition-kernels}).
The resource in question would be depletable or pre-emptable, in order for overlap of utilization functions to translate into competition. Pairwise competition coefficients (such as the $\alpha_{ij}$ parameters in Lotka-Volterra competition) could be calculated from the degree overlap between utilizations functions \citep{MacArthur-1967, May-1972, Abrams-1975}.
If a trait was used as a niche index, it was chosen on grounds of mapping tidily on to a depletable resource, such as beak size for seed-eating finches.
Beginning in the 1980s, a desire to model character displacement saw the concept of a utilisation function first approximated via a trait -- the mean of the utilisation function 
% maybe this is my poor english but the mean size of the utilisation function was not so clear for me (what is teh size of the utilisation function?).
\citep{Roughgarden-1979} -- and second replaced entirely by a function of trait differences \citep{Slatkin-1980,Taper-1985}. 
This shift enabled and flowed into models for character displacement \citep[e.g.,][]{Taper-1985, Case-2000, Goldberg-2006} and trait evolution \citep{Brown-1987,Geritz-1998,Geritz-1999,Dieckmann-1999,  Abrams-2001}.


Assumptions about the nature of competition also feature in an other related strand of thinking that persist to current day. In a break from the idea of subdividing continuous niche axes, models arose that visualised communities as a chessboard of living sites, each able to be occupied by one sessile individual \citep{Yodzis-1978}. Coexistence could arise by colonization-competition trade-offs \citep{Levins-1971, Geritz-1999, Calcagno-2006}; which required that competitive interactions be strongly size-asymmetric. Alternatively, if there is an advantage to already established individuals (another form of competitive asymmetry), chance successes in the colonisation of vacant sites  -- arising from demographic stochasticity (``lottery effects'') \citep{Sale-1977, Hubbell-2001} -- can slow competitive exclusion and thereby promote coexistence.


The important point here is that these models also make strong assumptions about the shape of the competition function being used -- a sharp logistic function in competition colonisation models and a flat, trait-independent function in neutral lottery models.

In all these avenues, the shape of the trait-based competition function is central to the question how species interact within the models using them.
By far the most common type of competition function in textbooks and in theoretical literature is that they are unimodal (Gaussian), decreasing with trait-distance away from the species that is exerting the competitive effect. This shape was key in the development of the concept of limiting similarity \citep{MacArthur-1967}.
Yet, other shapes are possible  and the menagerie of possible functions has only begun to be catalogued \citep{Barabas-2012,Leimar-2013} (Table \ref{tab:shapes}). In practice, researchers have focussed on a limited number of function shapes, in particular Gaussian, asymmetric Gaussian or logistic shapes, probably more because of mathematical tractability than empirical reality.

It has also become increasingly clear, that the shape of competition function is key to understanding community assembly. Already in the 1970s, and in opposition with the species packing metaphor, several papers questioned whether there was in fact a limit to similarity \citep{May-1972, Abrams-1975, Abrams-1983}, with debate continuing until recently \citep{Meszena-2006, Barabas-2012, Leimar-2013}.
Competitive exclusion can become slow because two competitors are similar in their resource use as well as because they are different. Since 2000, this line of theory began to explore more systematically the consequences of possible function shapes for community assembly, showing that subtle details about the exact shape of the function can substantially alter the possibilities for coexistence \citep{Calcagno-2006, Scheffer-2006, Pigolotti-2007, Barabas-2012,Barabas-2013,DAndrea-2013,Leimar-2013}.
Although this strand of theory might have been expected to undermine the mainstream approach to community ecology via the species packing metaphor, in actuality it did not have the effect of dislodging species packing from standard textbooks, probably because it did not really offer an alternative program for connecting theory to field measurement.
