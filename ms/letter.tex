
In the Comments field, please explain how this manuscript fits the goals of the journal or the specific section of the journal. The American Naturalist aims to publish papers that

- are of interest to the broad readership,
- pose a new and significant problem or introduce a novel subject,
- change the way people think about the topic of the manuscript, and/or
    confirm or refute an unverified theoretical principle or a previously unsupported or weakly supported generalization.


In this paper we attempt to bridge between two broad classes of model used in community ecology: phenomenological models based on pairwise species interactions, and those baed on 


Community ecology relies heavily on theoretical models that represent species interactions via a competition coefficient: a single number capturing the per capita effect of individuals from one species on the population growth rate another. A common way to generalise such coefficients beyond pairwise species combinations has been to them as functions of traits. Yet the evidence supporting this approach is limited. We assess the utility of trait-based competition functions by comparing their shapes to those emerging from models in which their is explicit competition for a shared resource. Our results show that the common shapes for competition functions used in phenomenological models bare little resemblance to those emerging from direct resource competition, and will thereby bring readers to question this previously unsupported generalization and approach.

